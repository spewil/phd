<!DOCTYPE html>
<html xmlns="http://www.w3.org/1999/xhtml" lang="" xml:lang="">
  <head>
    <meta charset="utf-8" />
    <meta name="generator" content="pandoc" />
    <meta name="viewport" content="width=device-width, initial-scale=1.0, user-scalable=yes" />
        <meta name="author" content="Spencer R. Wilson" />
                <title></title>
    <style>
    code{white-space: pre-wrap;}
    span.smallcaps{font-variant: small-caps;}
    span.underline{text-decoration: underline;}
    div.column{display: inline-block; vertical-align: top; width: 50%;}
    div.hanging-indent{margin-left: 1.5em; text-indent: -1.5em;}
    ul.task-list{list-style: none;}
    </style>
        <link rel="stylesheet" href="pandoc.css" />
            <!--[if lt IE 9]>
    <script src="//cdnjs.cloudflare.com/ajax/libs/html5shiv/3.7.3/html5shiv-printshiv.min.js"></script>
    <![endif]-->
        <script src="https://hypothes.is/embed.js" async></script>
      </head>
  <body>
      
            <header id="title-block-header">
        <h1 class="title">Sensorimotor Learning in Virtual
Environments</h1>
                        <p class="author">Spencer R. Wilson</p>
                        <!-- <p class="date">1/1/2021</p> -->
        <p class="date">Last updated: <script> document.write(new Date().toLocaleDateString());</script></p>

              </header>
      
            <div class="toc-col">
        <div class="toc">
          <nav id="TOC" role="doc-toc">
                        <h2 id="toc-title">Contents</h2>
                        
          </nav>
        </div>
      </div>
      
      <div class="main-col">
          <div class="main">
            Where are you?

            This is an experiment in creating an open kind of thesis. To
            start adding comments to this page, just highlight some
            text, click \texttt{annotate} and start typing. Note that
            you will have to a Hypothes.is account, but it only takes a
            moment (and it's a nonprofit organization). Add as many
            comments as you like!

            \hypertarget{sec:intro}{%
            \section{Introduction \& Aims}\label{sec:intro}}

            \begin{quote}
            \emph{Movement is nothing but the quality of our being.}

            --- Sunryu Suzuki
            \end{quote}

            Named after roboticist Hans Moravec, Moravec's Paradox
            states that it is easier to generate artificially
            intelligent performance on tasks we think of as
            intellectually challenging, such as chess, than to provide a
            machine with faculties we take for granted, such as
            movement. For example, Moravec's Paradox encourages us to
            not look past the stunningly complex computations generated
            by the human motor apparatus. Following Moravec's
            suggestion, this work focuses on the human motor system
            which, we argue, is the most advanced control system in the
            known universe.

            A recent review corroborates this perspective and provides a
            clear call to action:

            \begin{quote}
            The processes by which biological control solutions spanning
            large and continuous state spaces are constructed remain
            relatively unexplored. Future investigations may need to
            embed rich dynamical interactions between object dynamics
            and task goals in novel and complex
            movements.\textsuperscript{\protect\hyperlink{ref-McNamee2019}{1}}
            \end{quote}

            The generation of skilled movement is an open problem. The
            aim of this thesis is to progress our understanding of motor
            output by studying the solutions produced by human subjects
            to solve tasks in dynamically rich, yet controlled, virtual
            environments. Our goal is to reverse-engineer our shared
            ability to acquire novel motor skills. We must define what
            we mean by the terms \emph{skill}, and \emph{task}.

            Humans produce a great variety of movements every day, often
            without conscious thought. For example, movements like
            bringing a cup of coffee to our lips for a sip are generally
            out of reach for state-of-the-art robotic systems. We claim
            that this ``motor gap'' between biological and artificial
            motor systems is due to a lack of \emph{dexterity} in the
            latter. Soviet Neuroscientist Nikolai Bernstein defined
            dexterity as the ability to ``find a motor solution in any
            situation and in any condition.'' The crux of this
            definition is the flexibility of such solutions. This
            flexibility, or robustness\footnote{Kitano}, is the ability
            to optimize internal parameters in response to external
            perturbations and adapt to new information to achieve the
            goals of an ongoing plan.

            While a robot may be able to move a cup of coffee to a
            precise location in space, its solution is often found to be
            brittle in a new context, or unable to generalize to the
            movement of new objects. We define a skill as a behavior
            that involves dexterity in Bernstein's sense. The use of a
            tool such as a screwdriver is an example of a motor skill.
            We define a task as the production of skilled movement in a
            particular context. Driving a screw in a particular posture
            using a particular screwdriver is an example of a task.
            These concepts will be further formalized in later chapters.

            All human movement is the result of the activation and
            subsequent contraction of muscle fibers, and all movements
            lie along a continuum between reflexive and volitional. It
            is the supramuscular circuitry which determines the degree
            of volition we ascribe to movement. In this thesis, we chose
            to focus on movements of the human hand as the hand is a
            unique evolutionary invention that underlies our ability to
            various skills in a range of tasks. The hand is the pinnacle
            of dexterity. The hand highlights humanity's unique It could
            be argued that the hand is in fact crucial aspect of
            humanness itself.

            \begin{itemize}
            \tightlist
            \item
              experiments with interesting and novel movement data
            \item
              analysis of muscle and behavioral data
            \end{itemize}

            The central question of this thesis is the nature of the
            computational processes driving sequential selection of
            these contractions goal-oriented behavior.

            track learning of subjects over many sessions.

            \hypertarget{what-is-our-goal}{%
            \subsection{What is our goal?}\label{what-is-our-goal}}

            We seek to experimentally test models of decision-making.

            \begin{itemize}
            \tightlist
            \item
              inject RL / control theory into skill acquisition
            \item
              what do we mean by ``RL'' here? is ``RL'' relevant? why?
            \end{itemize}

            \hypertarget{deterity-what-is-unique-about-human-movement}{%
            \subsection{Deterity -- what is unique about human
            movement?}\label{deterity-what-is-unique-about-human-movement}}

            Our setup, described in \cref{sec:setup}, affords us the
            ability to track learning of a novel motor skill similar to
            the use of a new tool.

            The position taken in this thesis that volitional movements
            are generated in terms of policies or controllers.

            Loops -- at the level of reflexes, within the brain, within
            cortex. Even reflexes can be modualted by value-based.

            What strategies are used to explore this space of possible
            mappings between what you experience when you move and what
            you expect to see and feel as a result?

            Thus, the control system governing movement of the human
            hand is an ideal testbed for quantifying changes in muscle
            activity during skill acquisition.

            \hypertarget{section}{%
            \subsection{}\label{section}}

            \hypertarget{og-phd-proposal}{%
            \subsection{OG phd proposal}\label{og-phd-proposal}}

            \hypertarget{todorov-2009}{%
            \subsubsection{Todorov 2009}\label{todorov-2009}}

            From a 2009 review suggesting exactly the work that our
            hunch is leading us towards:

            \begin{quote}
            First, analyses such as that performed by Valero-Cuevas et
            al.~{[}42{]} and Kutch et al.~{[}40{]} should be done across
            many different behaviors and a wider range of behavioral
            conditions to evaluate whether the structure in the
            variability of muscle activation patterns is consistent with
            the muscle synergy hypothesis. Although the analyses used in
            those experiments exploit some ideal features of finger
            control, similar experiments should be possible in other
            behaviors and would help address concerns about synergies
            arising from task constraints. Second, it should be possible
            to use synergies to explain suboptimal performance of the
            CNS {[}70{]}. If the CNS has access to a limited set of
            synergies at a particular time based on the tasks that it
            currently is able to accomplish, this should suggest that
            some new tasks should be easier to perform than others {[}44
            {]}: if the muscle activation patterns required by the new
            task lay within the space defined by existing muscle
            synergies, learning the new task should be relatively easy.
            In contrast, if the required activations lay outside that
            space, then the learning should be more difficult and
            initial performance should be suboptimal. Designing such
            tasks requires an accurate musculoskeletal model along with
            knowledge of the existing muscle synergies which would make
            it possible to predict which tasks would be easy and which
            would be difficult to learn.
            \end{quote}

            Additionally, the review authors provide an argument for a
            developmental basis of the synergies we find in EMG
            recordings:

            \begin{quote}
            Rather than considering muscle synergies as reflecting a
            strategy for the simplification of control, we suggest that
            synergies might be considered in the larger context of the
            intimate interactions between the properties of the
            musculoskeletal system and neural control strategies. In
            this context, muscle synergies could be considered as
            reflecting the statistics of the external world,
            acknowledging the fact that the external world also consists
            of the musculoskeletal system itself. In the same way that
            properties of natural scenes might influence the structure
            of the visual system, we suggest that statistics of the
            musculoskeletal system and external world might influence
            the structure of motor systems.
            \end{quote}

            Note that the authors' second suggestion has been tested in
            a reaching task. The results concorded with the hypothesis
            from the quoted review, as we would expect:

            \begin{quote}
            After compatible virtual surgeries, a full range of
            movements could still be achieved recombining the synergies,
            whereas after incompatible virtual surgeries, new or
            modified synergies would be required. Adaptation rates after
            the two types of surgery were compared. If synergies were
            only a parsimonious description of the regularities in the
            muscle patterns generated by a nonmodular controller, we
            would expect adaptation rates to be similar, as both types
            of surgeries could be compensated with similar changes in
            the muscle patterns. In contrast, as predicted by
            modularity, we found strikingly faster adaptation after
            compatible surgeries than after incompatible ones.
            \end{quote}

            However, seeing that the mapping between the recorded EMG
            and the output was a multilinear regression based on a
            calibration dataset which was grossly altered for the
            surgery, I do not find it surprising that the learning
            curves were different after the two surgeries.

            \hypertarget{bibliography}{%
            \subsection*{Bibliography}\label{bibliography}}
            \addcontentsline{toc}{subsection}{Bibliography}

            \hypertarget{refs}{}
            \begin{CSLReferences}{0}{0}
            \leavevmode\hypertarget{ref-McNamee2019}{}%
            \CSLLeftMargin{1. }
            \CSLRightInline{McNamee, D. \& Wolpert, D. M. Internal
            {Models} in {Biological Control}. \emph{Annual Review of
            Control, Robotics, and Autonomous Systems} \textbf{2},
            339--364 (2019).}

            \end{CSLReferences}
                      </div>
      </div>


  </body>
</html>