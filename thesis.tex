<!DOCTYPE html>
<html xmlns="http://www.w3.org/1999/xhtml" lang="" xml:lang="">
  <head>
    <meta charset="utf-8" />
    <meta name="generator" content="pandoc" />
    <meta name="viewport" content="width=device-width, initial-scale=1.0, user-scalable=yes" />
        <meta name="author" content="Spencer R. Wilson" />
                <title></title>
    <style>
    code{white-space: pre-wrap;}
    span.smallcaps{font-variant: small-caps;}
    span.underline{text-decoration: underline;}
    div.column{display: inline-block; vertical-align: top; width: 50%;}
    div.hanging-indent{margin-left: 1.5em; text-indent: -1.5em;}
    ul.task-list{list-style: none;}
    </style>
        <link rel="stylesheet" href="pandoc.css" />
            <!--[if lt IE 9]>
    <script src="//cdnjs.cloudflare.com/ajax/libs/html5shiv/3.7.3/html5shiv-printshiv.min.js"></script>
    <![endif]-->
        <script src="https://hypothes.is/embed.js" async></script>
      </head>
  <body>
      
            <header id="title-block-header">
        <h1 class="title">Sensorimotor Learning in Virtual
Environments</h1>
                        <p class="author">Spencer R. Wilson</p>
                        <!-- <p class="date">1/1/2021</p> -->
        <p class="date">Last updated: <script> document.write(new Date().toLocaleDateString());</script></p>

              </header>
      
            <div class="toc-col">
        <div class="toc">
          <nav id="TOC" role="doc-toc">
                        <h2 id="toc-title">Contents</h2>
                        
          </nav>
        </div>
      </div>
      
      <div class="main-col">
          <div class="main">
            Where are you?

            This is an experiment in creating an open kind of thesis. To
            start adding comments to this page, just highlight some
            text, click \texttt{annotate} and start typing. Note that
            you will have to a Hypothes.is account, but it only takes a
            moment (and it's a nonprofit organization). Add as many
            comments as you like!

            \hypertarget{bibliography}{%
            \subsection{Bibliography}\label{bibliography}}
                      </div>
      </div>


  </body>
</html>