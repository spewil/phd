<!DOCTYPE html>
<html xmlns="http://www.w3.org/1999/xhtml" lang="" xml:lang="">
  <head>
    <meta charset="utf-8" />
    <meta name="generator" content="pandoc" />
    <meta name="viewport" content="width=device-width, initial-scale=1.0, user-scalable=yes" />
        <meta name="author" content="Spencer R. Wilson" />
                <title></title>
    <style>
    code{white-space: pre-wrap;}
    span.smallcaps{font-variant: small-caps;}
    span.underline{text-decoration: underline;}
    div.column{display: inline-block; vertical-align: top; width: 50%;}
    div.hanging-indent{margin-left: 1.5em; text-indent: -1.5em;}
    ul.task-list{list-style: none;}
    </style>
        <link rel="stylesheet" href="pandoc.css" />
            <!--[if lt IE 9]>
    <script src="//cdnjs.cloudflare.com/ajax/libs/html5shiv/3.7.3/html5shiv-printshiv.min.js"></script>
    <![endif]-->
        <script src="https://hypothes.is/embed.js" async></script>
      </head>
  <body>
      
            <header id="title-block-header">
        <h1 class="title">Sensorimotor Learning in Virtual
Environments</h1>
                        <p class="author">Spencer R. Wilson</p>
                        <!-- <p class="date">1/1/2021</p> -->
        <p class="date">Last updated: <script> document.write(new Date().toLocaleDateString());</script></p>

              </header>
      
            <div class="toc-col">
        <div class="toc">
          <nav id="TOC" role="doc-toc">
                        <h2 id="toc-title">Contents</h2>
                        
          </nav>
        </div>
      </div>
      
      <div class="main-col">
          <div class="main">
            Where are you?

            This is an experiment in creating an open kind of thesis. To
            start adding comments to this page, just highlight some
            text, click \texttt{annotate} and start typing. Note that
            you will have to a Hypothes.is account, but it only takes a
            moment (and it's a nonprofit organization). Add as many
            comments as you like!

            \hypertarget{sec:intro}{%
            \section{Introduction \& Aims}\label{sec:intro}}

            \begin{itemize}
            \tightlist
            \item
              what are we interested in?

              \begin{itemize}
              \tightlist
              \item
                why can't robots move like humans?
              \item
                What is special about human movement?
              \end{itemize}
            \item
              why is this the most interesting problem?
            \item
              how are we going to approach this interest?
            \item
              what do we hope to achieve?

              \begin{itemize}
              \tightlist
              \item
                experiments with interesting and novel movement data
              \item
                analysis of muscle and behavioral data
              \item
                inject RL / control theory into skill acquisition

                \begin{itemize}
                \tightlist
                \item
                  what do we mean by ``RL'' here?
                \item
                  is ``RL'' relevant? why?
                \end{itemize}
              \end{itemize}
            \end{itemize}

            Named after roboticist Hans Moravec, the Moravec Paradox
            states that it is easier to generate artificially
            intelligent performance on tasks we think of as
            intellectually challenging, such as chess, than to provide a
            machine with faculties we take for granted, such as
            movement. Simply put, the human motor apparatus is the most
            advanced control system in the known universe. Tasks such as
            bringing a glass of water to your lips are referred to in
            the literature as a ``problem'' despite being effortless for
            the majority of people. A recent review provides a clear
            call to action:

            \begin{quote}
            The processes by which biological control solutions spanning
            large and continuous state spaces are constructed remain
            relatively unexplored. Future investigations may need to
            embed rich dynamical interactions between object dynamics
            and task goals in novel and complex
            movements.\textsuperscript{\protect\hyperlink{ref-McNamee2019}{1}}
            \end{quote}

            This thesis attempts to make progress on advancing our
            understanding of naturalistic movement by studying solutions
            to motor problems produced by human subjects in controlled
            experiments. These experiments leverage a real-time,
            high-dimensional electromyography (EMG) pipeline to create
            closed-loop virtual motor learning environments.

            Due to the unique physiology governing the control of the
            human hand\cref{sec:physiology}.

            reference\cref{sec:intro}

            Loops -- at the level of reflexes, within the brain, within
            cortex. Even reflexes can be modualted by value-based.

            The problem is one of decision-making, as studied by
            cognitive neuroscience. We wish to extend such thinking into
            the problem of volitional, goal-directed motor output
            selection. Deciding, that is, the sequence of muscle
            activations to produce given a current perceived state and a
            goal.

            We know that it is involved in these muscle contractions,
            but what sort of strategy do you use to explore this space
            of possible mappings between what you experience when you
            move and what you expect to see and feel as a result? This
            is the question I hope to make headway on.

            To do this, I'll use the literature of reinforcement
            learning and optimal control theory to guide my theoretical
            understanding of what is happening when a subject begins to
            experience learning in this novel situation. I will model
            hypotheses of this learning process and compare these models
            to the large amounts of data my experimental setup will
            produce as we track learning of subjects over many sessions.

            For my PhD project, I propose developing a real-time,
            high-dimensional electromyography (EMG) pipeline to create
            closed-loop virtual motor learning experiments with human
            subjects which involve tasks with precisely this kind of
            dynamical richness.

            We aim to build on our current understanding and models of
            continuous control in humans with an eye towards
            illustrating how the variability in our motor output evolves
            over learning a novel, highly-skilled task.

            The human hand is a unique evolutionary invention that
            enables an unprecedented ability to manipulate objects in a
            range of tasks. Recent work has shown that monosynaptic
            cortical projections controlling the digits, the
            corticomotoneuronal (CM) tract, act in coordination with
            synergistic muscle activations of the hand to achieve
            control that is balanced between modularity and
            flexibility{[}20, 21, 27{]}. These findings suggest that
            this bipartite structure in human motor cortex driving
            dexterous control of the distal part of the upper limb is
            due to evolutionary pressure to quickly generalize between
            tasks.

            Thus, the control system governing movement of the human
            hand is an ideal testbed for quantifying changes in muscle
            activity during skill acquisition.

            Classical laboratory tasks, such as reaching under
            perturbations, tend not to reflect the statistical richness
            of natural sensorimotor control and learning. Natural
            learning processes unfold across multiple timescales, and
            humans have a unique ability to quickly optimize internal
            parameters in response to external perturbations and new
            sensory information to achieve the goals of an ongoing plan.

            To engineer this flexible learning in silico, we must
            understand how humans adapt to a novel sensorimotor mapping.
            There exist a handful of prior studies mapping EMG activity
            and finger joint angles directly to virtual stimuli, though
            few are focused on the learning process and none have the
            input dimensionality we aim to achieve in work proposed
            here{[}3, 16, 19, 6, 14{]}.

            In one sense, our goal is to reverse-engineer our ability to
            acquire novel motor skills. This will require three
            sequential phases: characterizing the space of naturalistic
            motor behaviors recorded via surface EMG, determining the
            ability of healthy subjects to perform tasks outside of this
            space of naturalistic muscle activity, and modeling the
            learning process for tasks designed with knowledge from the
            first two phases.

            \hypertarget{synergies}{%
            \subsection{Synergies}\label{synergies}}

            A considerable amount of research has focused on the
            existence of synergies as a simplifying structure in the
            motor system. We believe that the concept of synergies is
            often attributed to the process of motor control as opposed
            to a strict structural constraint. In this work, we use a
            bespoke experimental setup to track spatiotemporal dynamics
            of synergistic muscle activations across learning in a
            virtual, high-dimensional, electromyographic-driven task
            involving muscle contractions of the hand and forearm. We
            find that over trials the motor system adapts its
            synergistic action to fulfill the predefined task
            requirements in an optimal manner.

            \hypertarget{bmi}{%
            \subsection{BMI}\label{bmi}}

            \hypertarget{arbitrary-visuomotor-mappings}{%
            \subsection{Arbitrary Visuomotor
            Mappings}\label{arbitrary-visuomotor-mappings}}

            There are several studies using non-EMG-driven sensorimotor
            mappings to study human motor control and learning.

            x * Remapping Hand Movements in a Novel Geometrical
            Environment https://www.ncbi.nlm.nih.gov/pubmed/16148276

            vocoder machine bell labs

            Hinton, Fells

            palsy study

            takehome: humans are really good at learning tasks like
            these, especially with their hands. this type of dexterity
            is specific to primates if not humans. let's use this
            ability to understand and try to model how this learning
            process unfolds.

            \textbf{\emph{What does this give us that a force-field
            reaching task can't?}}

            \hypertarget{muscle-synergies}{%
            \subsection{Muscle Synergies}\label{muscle-synergies}}

            \begin{itemize}
            \tightlist
            \item
              Neural basis for hand muscle synergies in the primate
              spinal cord https://www.ncbi.nlm.nih.gov/pubmed/28739958
            \end{itemize}

            Two mutually non-exclusive scenarios can be envisioned as to
            how corticospinal (and reticulospinal --see Baker, 2011,
            this issue) pathways might be organized to coordinate the
            activities of multiple muscles needed to perform finger
            movements (Schieber, 1990). In one,separate pathways operate
            on each of the requisite motor nuclei. In the other,
            selection of the muscles into functional groups is
            determined in part by the pattern of divergence of
            individual descending pathways across different motor nuclei
            in the spinal cord. This latter type of organization,while
            less flexible, might underlie the assemblage of muscles into
            synergistic groups that serve as the building blocks of the
            behavioural repertoire of an animal. In contrast to the
            extrinsic muscles of the dominant hand described above,
            virtually no short-term synchrony was observed across
            intrinsic muscles participating in the precision grip
            (McIsaac \& Fuglevand, 2008). This result suggests that the
            descending pathways that control the activities of intrinsic
            muscles provide more concentrated input to individual motor
            nuclei than those pathways destined for motor nuclei
            innervating extrinsic hand muscles. The contrasting
            organizations of the descending pathways targeting extrinsic
            and intrinsic muscles seem in harmony with postulated
            functions of these two groups of muscles(Longet al.1970).
            Intrinsic muscles configure the digits to the unique
            dimensions of an object to be handled. HighlyFigure 5. Mean
            (SD) common input strength (CIS -- index representing
            magnitude of short-term synchrony; Nordstrometal.1992) for
            pairs of motor units residing in the same compartment or
            adjacent compartments of three human multi-tendoned hand
            muscles, extensor digitorum (ED), flexor digitorum
            superficialis (FDS) and flexor digitorum profundus(FDP)Mean
            (SD) CIS values: ED same=0.70 (0.30), ED
            adjacent=0.41(0.18), FDS same=0.45 (0.30), FDS adjacent=0.27
            (0.17), FDPsame=0.47 (0.19), FDP adjacent=0.36 (0.21).
            Values inside ofbars indicate number of motor unit pairs.
            Data compiled from:†Keen \& Fuglevand (2004b); McIsaac \&
            Fuglevand (2007); McIsaac\& Fuglevand (unpublished data);
            Winges \& Santello (2004).independent pathways, therefore,
            enable the fractionated actions of the digits needed for
            such a function. Extrinsic muscles provide the primary
            gripping forces during object manipulation. Because gripping
            necessitates the production of precisely counterbalanced
            forces between the thumb and one or more fingers, extrinsic
            muscles have their activities linked by divergent descending
            inputs. (Fuglevand 2011)

            \hypertarget{precision-grip}{%
            \subsubsection{Precision Grip}\label{precision-grip}}

            Results imply that descending pathways diverge extensively
            to operate on the two motor nuclei supplying thumb and index
            finger muscles as a unit and thereby compel them to operate
            in unison. Interestingly, such across-muscle synchrony was
            seen only in the dominant but not in the non-dominant hand
            (Fig. 4). Whether such lateralized differences are laid down
            early in development or represent plastic changes associated
            with chronic usage are questions currently under
            investigation. (Fuglevand 2011)

            Statistics of Natural Movements Are Reflected in Motor
            Errors (Wolpert)
            https://www.ncbi.nlm.nih.gov/pubmed/19605616

            Structural Learning, Wolpert+Braun+Mehring
            https://www.ncbi.nlm.nih.gov/pmc/articles/PMC2692080/

            \hypertarget{skilled-piano-performance}{%
            \subsubsection{Skilled Piano
            Performance}\label{skilled-piano-performance}}

            In piano performance, for keystrokes with each of the four
            fingers during playing various tone sequences, the hand
            kinematics was characterized by three distinct patterns of
            finger joint coordination (Furuya et al., 2011a). The motion
            of the striking finger was consistent across these patterns,
            whereas the motion of the non-striking fingers differed
            across them. This was interpreted as evidence for the
            independence of movements across fingers. In addition, the
            amount of movement covariation between the striking and
            non-striking fingers was similar, independent of which
            finger was used for a keystroke. The finding was in contrast
            to non-musicians who displayed a hierarchy of independence
            of finger movements, the middle and ring fingers being less
            individuated than the index and little fingers (Häger-Ross
            and Schieber, 2000; Zatsiorsky et al., 2000). The equal
            independence of movements across fingers can be therefore
            achieved by extensive piano training. This idea is supported
            by superior independence of finger movement control for
            pianists as compared to non-musicians (Slobounov et al.,
            2002; Aoki et al., 2005), which possibly occurs due to
            changes at biomechanical and neural levels (Chiang et al.,
            2004; Smahel and Klimová, 2004). (Shinichi Furuya* and
            Eckart Altenmüller 2013)

            In piano performance, not all digits necessarily move for
            the production of a tone. Depending on contexts and task
            demands, some digits either move anticipatorily to
            facilitate production of upcoming acoustic events or even do
            not have to move. The former anticipatory modification of
            the movements is called coarticulation and serves as a
            mechanism that ensures smooth succession of sequential
            movements such as speech (Ostry et al., 1996) and finger
            spelling (Jerde et al., 2003). This coarticulation was also
            evident in piano playing, particularly when the hand posture
            changes dynamically (Engel et al., 1997). For example, the
            fingers and wrist initiated preparatory motions 500 ms prior
            to the thumb-under maneuver, which facilitated the
            subsequent horizontal translation of the hand. Finger
            muscular activity also provided evidence supportive for
            co-articulation in piano playing (Winges et al., 2013). The
            balance of burst amplitudes across multiple muscles depended
            on the characteristics of the preceding and subsequent
            keypresses, forming neuromuscular co-articulation throughout
            the time course of sequential finger movements. (Shinichi
            Furuya* and Eckart Altenmüller 2013)

            \hypertarget{muscular-and-postural-synergies-of-the-human-hand-weiss-flanders-2004}{%
            \subsubsection{Muscular and Postural Synergies of the Human
            Hand (Weiss \& Flanders
            2004)}\label{muscular-and-postural-synergies-of-the-human-hand-weiss-flanders-2004}}

            \begin{quote}
            Single motor units receive a variety of motor commands, and
            the net result may be that neighboring units in the same
            muscle are preferentially recruited to produce forces in
            different directions 3D space or to hold the hand in
            different static postures. The corollary to this is that a
            given force or posture involves a collection of units that
            spans many muscles.
            \end{quote}

            The motor system is distributed in order to solve the
            redundancy problem as well as to learn new control schemes.

            \hypertarget{analysis-of-the-synergies-underlying-complex-hand-manipulation-todorov-ghahramani-2005-blah-blah-blah-blah-blah-blah}{%
            \subsubsection{Analysis of the synergies underlying complex
            hand manipulation (Todorov \& Ghahramani 2005) blah blah
            blah blah blah
            blah}\label{analysis-of-the-synergies-underlying-complex-hand-manipulation-todorov-ghahramani-2005-blah-blah-blah-blah-blah-blah}}

            \begin{quote}
            Remarkably, the dimensionality in the individuated joint
            task was 8.7, or only 2 higher. The latter task is designed
            to reveal the maximal number of degrees of freedom humans
            have access to. Why this number is not 20 is unclear; the
            most likely reason is biomechanical coupling, although
            limitations in neural control may also play a role.
            Furthermore, the number 8.7 intuitively seems too low ñ
            suggesting that such counting methods may underestimate the
            true dimensionality.
            \end{quote}

            \hypertarget{the-statistics-of-natural-hand-movements-ingram-wolpert-2008}{%
            \subsubsection{The statistics of natural hand movements
            (Ingram \& Wolpert
            2008)}\label{the-statistics-of-natural-hand-movements-ingram-wolpert-2008}}

            \hypertarget{neural-basis-of-muscle-synergies-bizzi-cheung-2013}{%
            \subsubsection{Neural Basis of Muscle Synergies, Bizzi
            Cheung
            2013}\label{neural-basis-of-muscle-synergies-bizzi-cheung-2013}}

            Note that there are a great number of tasks, and the case
            for synergies in the motor system cannot be answered simply.
            Here we are concerned with motions of the hand because we
            know that this is the endpoint of CM connections. There are
            many fewer tasks dealing with this system in particular.
            Most tasks deal with arm reaching, though the most highly
            cited synergy paper deals with a 1DOF
            kick.\textsuperscript{\protect\hyperlink{ref-DAvella2003}{2}}

            \hypertarget{todorov-2009}{%
            \subsubsection{Todorov 2009}\label{todorov-2009}}

            From a 2009 review suggesting exactly the work that our
            hunch is leading us towards:

            \begin{quote}
            First, analyses such as that performed by Valero-Cuevas et
            al.~{[}42{]} and Kutch et al.~{[}40{]} should be done across
            many different behaviors and a wider range of behavioral
            conditions to evaluate whether the structure in the
            variability of muscle activation patterns is consistent with
            the muscle synergy hypothesis. Although the analyses used in
            those experiments exploit some ideal features of finger
            control, similar experiments should be possible in other
            behaviors and would help address concerns about synergies
            arising from task constraints. Second, it should be possible
            to use synergies to explain suboptimal performance of the
            CNS {[}70{]}. If the CNS has access to a limited set of
            synergies at a particular time based on the tasks that it
            currently is able to accomplish, this should suggest that
            some new tasks should be easier to perform than others {[}44
            {]}: if the muscle activation patterns required by the new
            task lay within the space defined by existing muscle
            synergies, learning the new task should be relatively easy.
            In contrast, if the required activations lay outside that
            space, then the learning should be more difficult and
            initial performance should be suboptimal. Designing such
            tasks requires an accurate musculoskeletal model along with
            knowledge of the existing muscle synergies which would make
            it possible to predict which tasks would be easy and which
            would be difficult to learn.
            \end{quote}

            Additionally, the review authors provide an argument for a
            developmental basis of the synergies we find in EMG
            recordings:

            \begin{quote}
            Rather than considering muscle synergies as reflecting a
            strategy for the simplification of control, we suggest that
            synergies might be considered in the larger context of the
            intimate interactions between the properties of the
            musculoskeletal system and neural control strategies. In
            this context, muscle synergies could be considered as
            reflecting the statistics of the external world,
            acknowledging the fact that the external world also consists
            of the musculoskeletal system itself. In the same way that
            properties of natural scenes might influence the structure
            of the visual system, we suggest that statistics of the
            musculoskeletal system and external world might influence
            the structure of motor systems.
            \end{quote}

            Note that the authors' second suggestion has been tested in
            a reaching task. The results concorded with the hypothesis
            from the quoted review, as we would expect:

            \begin{quote}
            After compatible virtual surgeries, a full range of
            movements could still be achieved recombining the synergies,
            whereas after incompatible virtual surgeries, new or
            modified synergies would be required. Adaptation rates after
            the two types of surgery were compared. If synergies were
            only a parsimonious description of the regularities in the
            muscle patterns generated by a nonmodular controller, we
            would expect adaptation rates to be similar, as both types
            of surgeries could be compensated with similar changes in
            the muscle patterns. In contrast, as predicted by
            modularity, we found strikingly faster adaptation after
            compatible surgeries than after incompatible ones.
            \end{quote}

            However, seeing that the mapping between the recorded EMG
            and the output was a multilinear regression based on a
            calibration dataset which was grossly altered for the
            surgery, I do not find it surprising that the learning
            curves were different after the two surgeries.

            \hypertarget{sec:physiology}{%
            \section{Physiology of the Motor
            System}\label{sec:physiology}}

            \hypertarget{corticomotoneuronal-connections}{%
            \subsection{Corticomotoneuronal
            Connections}\label{corticomotoneuronal-connections}}

            \begin{itemize}
            \tightlist
            \item
              Subdivisions of primary motor cortex based on
              cortico-motoneuronal cells (Rathelot, Strick PNAS 2008)
              https://www.pnas.org/content/106/3/918
            \end{itemize}

            x Corticomotoneuronal cells are ``functionally tuned''
            https://www.ncbi.nlm.nih.gov/pmc/articles/PMC4829105/

            Although CM synapses exert powerful excitatory
            effectsonα-motoneurons, the same motoneurons also
            receivemany other inputs (Fuglevand, 2011), including
            thosefrom Ia afferents, spinal interneurons,
            propriospinalneurons, reticulospinal neurons and rubrospinal
            neurons(Mewes \& Cheney, 1991; Flamentet al.1992; Porter\&
            Lemon, 1993; Perlmutteret al.1998; Riddleet al.2009). Any
            individualα-motoneuron receives
            thousandsofsynapticcontacts.Thehandfulofsynapticcontactsfroma
            given CM cell therefore does not consistently drive
            thedischarge of any particularα-motoneuron. (Schieber 2011)

            rathelot, strick?

            \begin{itemize}
            \item
              ``Following a century of detailed anatomical
              tract-tracing, electrophysiological investigation and
              careful lesion studies, our knowledge of the executive
              pathways through which `commands' for movement must pass
              is unrivalled, yet we are still some way from really
              understanding how a movement is generated, which
              structures and pathways are involved and how they interact
              during the period leading up to movement onset.''
            \item
              The intense, synchronised output generated by electrical
              stimulation of macaque motor cortex evokes responses in
              hand and digit muscles with onset latencies of only
              approximately 10ms. These latencies are much shorter than
              the 60--100 ms between the onset of changes in M1 activity
              and the onset of muscle activity during voluntarily
              generated movements (Cheney and Fetz, 1980; Porter and
              Lemon, 1993).
            \item
              The discovery that M1 neurons can become active during
              observation of the actions of others, but without any
              overt signs of movement in the observer (Vigneswaran et
              al., 2013) adds to a long list of evidence that motor
              cortex can be active in a number of different states, all
              of which are quite distinct from movement itself
              (Schieber, 2011). These include preparation for movement
              (Shenoy et al., 2013), mental rehearsal and imagination
              (Cisek and Kalaska, 2004; Dushanova and Donoghue, 2010;
              Macuga and Frey, 2012).
            \item
              PTNs exhibit additional features that are consistent with
              a role as `command' neurons. These include the fact that
              they make many collaterals to important subcortical motor
              structures, such as the red nucleus and the pontine nuclei
              (Ugolini and Kuypers, 1986), thereby providing `efference
              copy' of `commands' to the cerebellum.
            \item
              A further feature of some primate PTNs is that they make
              direct cortico-motoneuronal (CM) connections to alpha
              motoneurons (Porter and Lemon, 1993; Rathelot and Strick,
              2006; Zinger et al., 2013), allowing the motor cortex
              direct access to spinal motoneurons. Of course, the CM
              system does not act alone, but in parallel with many other
              local interneuronal mechanisms and other descending
              pathways (Baker, 2011; Lemon, 2008), seg- mental (Takei
              and Seki, 2010) and propriospinal systems (Kinoshita et
              al., 2012). It is interesting that CM synapses on
              motoneurons are not subject to presynaptic inhibition
              (Jackson et al., 2006), suggesting that other systems
              (e.g.~peripheral affer- ent inputs from the moving limb)
              do not use this mechanism to modify CM inputs to
              motoneurons, which would mean that information delivered
              by CM projections is allowed unfettered influence over
              target motoneurons.
            \item
              CM cells show changes in activity long before movement
              onset, far longer than the esti- mated conduction delays:
              thus CM cells exert their influence long before movement
              starts, but at a level that is subthreshold for motoneuron
              activation and subsequent movement.
            \item
              \textbf{CM cells are active for a whole range of different
              limb move- ments, including reach-to-grasp, precision grip
              and during tool- use by macaques (McKiernan et al., 1998;
              Muir and Lemon, 1983; Quallo et al., 2012). CM connections
              are particularly well- developed in primates with a high
              level of dexterity and who use tools. Interestingly, CM
              cells are characterised by showing not only increases, but
              also decreases in activity before and during precision
              movements (Maier et al., 1993; Quallo et al., 2012).
              Indeed, one way that M1 appears to control the pattern of
              muscle activity during grasp is to `disfacilitate' the
              excitatory drive to motoneurons.} (Lemon Kraskov 2019)
            \item
              The primate corticospinal tract shows many interesting
              features that distinguish it rather sharply from the
              rodent pathway. These include the size and dis- tribution
              of fibres within the tract, with a small but probably very
              significant population of fast-conducting axons (Firmin et
              al., 2014).
            \end{itemize}

            \hypertarget{synergies}{%
            \subsection{Synergies}\label{synergies}}

            \begin{quote}
            There are many variations on the motor synergy concept61;
            here we mean functional couplings of different joints or
            muscles such that motor control operates at the level of
            multi-joint coordination patterns ratherthan through
            independent control of all joints. Producing actionsat this
            slightly higher level of abstraction can facilitate
            explorationand learning of new skills as well as simplify
            planning. (Wayne Nature Comms Hierarchical Motor Control)
            \end{quote}

            There is a longstanding debate about the origins of muscle
            synergies that strongly mirrors the nature-nuture debate.
            Are synergies learned or are they hardwired? If they're
            hardwired, what physiological subsystem contains this
            hardwiring? We don't need to take a side because there is
            clear evidence that humans overcome synergies to adapt their
            motor outputs to solve novel tasks and overcome all types of
            changes in the motor loop (injury, fatigue, prism goggles,
            etc.) via well-studied (in laboratory tasks at least)
            adaptation mechanisms {[}Helmholtz, Wolpert, Todorov, newer
            work on synergy shifts such
            as.\textsuperscript{\protect\hyperlink{ref-DeRugy2012}{3}}
            The more interesting questions ask on what timescales and by
            what mechanisms does learning occur, and can we reverse
            engineer paradigms and tasks that improve the learning rate.

            \emph{There are very few tasks dealing with the hand in
            particular. What type of task would test the hypothesis that
            CM connections act to fractionate synergies of the hand such
            that we can tune a parameter of the task to require more or
            less influence of these direct connections? We would like to
            ask a user to fractionate synergies of the hand to different
            levels.}

            This requires first mapping the intrinsic available dynamics
            of the hand per user.

            We then would like to present fixed mappings between hand
            output (either through direct muscle activity or through a
            controller such as a force pad).

            There is work suggesting that CM connections synapse
            primarily on low threshold motor units that are recruited
            first. This would imply a difference in synergy
            fractionation at lower force as opposed to higher force.
            This can be tested by adding a force parameter within a
            task.

            This hunch was bolstered specifically by the work of Takei
            et al.~in their 2017 PNAS paper:

            \begin{quote}
            It is generally believed that the direct corticomotoneuronal
            (CM) pathway, which is a phylogenetically newer pathway in
            higher primates, plays a critical role in the fractionation
            of muscle ac- tivity during dexterous hand movements.
            However, the present study demonstrated that PreM-INs, which
            are phylogenetically older, have spatiotemporal properties
            that correlate with muscle synergies during voluntary hand
            movements. Therefore, it is likely that these two systems
            have specialized functions for the control of primate hand
            movements, namely ``fractionated control'' and ``synergistic
            control,'' respectively. The interaction of these two
            putative control systems might be the source of the
            exceptional versatility of primate hand move- ments. For
            example, a power grip (e.g., gripping a hammer) is
            characterized by the predominant coactivation of hand
            muscles. It is known that power grip requires less
            involvement of the CM system, and therefore might result
            more from the PreM-IN system. Conversely, fine control of
            individual finger movements (e.g., control of a fingertip
            force of a single digit) requires higher fractionation of
            individual muscles and probably depends more on the CM
            system. Indeed, muscle synergies are not active during fine
            individual finger movements in some cases. Precision grip
            requires the fractionation of hand muscles as well as their
            coactivation, and thus might depend on cooperation of both
            the CM and PreM-IN systems. These examples suggest that the
            optimal balance of the two control systems may vary
            according to task requirements. Optimization of balanced
            control may be an important factor also for the acquisition
            of new motor skills. For example, Berger et al.~demonstrated
            that learning a new movement that is compatible with
            existing muscle synergies occurs much more quickly than
            learning a movement requiring new muscle synergies. This
            implies that establishing, modifying, or masking muscle
            synergies requires more training. This might explain our
            everyday experience that highly fractionated movements
            require extensive practice (e.g., using chopsticks requires
            more extensive training than using spoons). This conceptual
            framework of balanced control systems may help future
            studies to clarify how our nervous system controls and
            acquires versatile hand
            functions.\textsuperscript{\protect\hyperlink{ref-Takei2017}{4}}
            \end{quote}

            This notion of an ``old'' and ``new'' motor cortex is not
            conceptual, but has been shown using viral tracing
            techniques.\textsuperscript{\protect\hyperlink{ref-Rathelot2009}{5}}

            As I see it, the goal is to build a falsifiable model which
            takes into account the bipartite structure of M1 into
            account, and find tasks that ostensibly require the direct
            descending connections to fractionate learned synergies. In
            effect, the hypothesis to test is that CM connections
            override the ``consolidated'' patterns putatively generated
            via spinal circuitry.

            Thus, this is a learning question, an experimental problem,
            and a modeling task rolled into one. We have a good hunch
            that is backed up by solid work. The question comes down to
            how much we can learn by recording as much muscle activity
            as we can and designing very clever tasks to test very
            clever models.

            \hypertarget{porter-lemon}{%
            \subsubsection{Porter \& Lemon}\label{porter-lemon}}

            It is clear that humans have the capacity to learn single
            motor unit (SMU) activation of the digits. Given what we
            know about voluntary movement and human digit control it is
            likely that this activation is under motor cortical control.
            This ability is somewhat surprising for two reasons:

            \begin{enumerate}
            \def\labelenumi{\arabic{enumi}.}
            \tightlist
            \item
              Individual corticomotoneurons contact multiple motor
              pools, and rarely (if ever) individual motor neurons.
            \item
              There is no physiological reason for the nervous system to
              have the ability to control individual muscles. The CNS
              (motor cortex in particular) is more likely to generate
              ``complex'' movements requiring simultaneous control of
              multiple synergistic muscle groups.
            \end{enumerate}

            Given the fact that most CM cells appear to facilitate
            activity in several muscles, inhibition would appear to be
            essential for the recruitment of an individual muscle
            without any concurrent activity in its functional or
            anatomical synergists. But how is this inhibition recruited
            and maintained during and after learning?

            I see two opportunities with this project. The first is
            practical: if we can generate definitive experimental
            evidence and a working theory for what is involved in
            voluntary control of single MUs in the forearm, we can build
            a ground-truth dataset for spike waveforms of various
            muscles under various conditions. The second is
            experimental: we can manipulate pertinent types of sensory
            feedback with the goal of answering interesting questions
            about the interplay between proprioception,
            agonist-antagonist interaction, and Renshaw cell
            functionality while building an understanding of how
            dynamics can be shifted for voluntary action to reverse
            engineer a learning process from the cellular level.

            I've written below a little more background and detail on
            what I am thinking about, and would welcome your feedback
            and critique.

            insights from studying the corticospinal tract

            The brain seems a thoroughfare for nerve-action passing on
            its way to the motor animal. It has been remarked that
            Life's aim is an act, not a thought. Today the dictum must
            be modified to admit that, often, to refrain from an act is
            no less an act than to commit one, because inhibition is
            co-equally with excitation a nervous activity. (Sherrington,
            Rede Lecture, 1933)

            Since we're working with a voluntary movement ostensibly of
            the digits, we know that we're going to be dealing with the
            corticospinal tract. I discovered a book called The
            Corticospinal Tract and Voluntary Movement by Porter and
            Lemon (1995) -- it's awesome. I highly recommend it,
            especially Chapter 4. The book focuses mainly on the hand,
            and Roger Lemon seems to be a living legend.

            The key insights from P\&L:

            The spinal cord's neuronal organization is based on
            relatively rigid muscular synergies, and a mechanism to
            fractionate this is of particular importance for the muscles
            of the hands and digits which may need to be employed in a
            variety of flexible associations during voluntary movements.

            {[}Obviously{]} the structure of corticospinal connectivity
            places a constraint on the number of possible output
            arrangements that can be employed by the cortex\ldots{}

            We can conclude that the fundamental organizing principle of
            the cortico-motoneuronal output is one of influence over
            activity in multiple muscles.

            {[}Fetz 1980{]} found that 67 percent of their
            corticomotoneuronal cells contacted more than 2 of the 5--6
            different muscles sampled, and that the mean number of
            muscles facilitated per cortico-motoneuronal cell was 2.4.
            {[}\ldots{]} The majority of CM neurones (59 per cent)
            produced pure facilitation of either an agonist or
            antagonist, and cofacilitation of both was extremely rare.
            The second most common pattern (30 percent of CM cells) was
            facilitation of agonists and postspike suppression of
            antagonists.

            Findings suggest that CM cell axons branch to innervate many
            (if not all) motor units within the motor nucleus of its
            target muscle.

            The widespread synaptic impacts made within the motor
            nucleus of a single muscle would allow a single CM cell to
            exert a facilitatory influence over many different
            motoneurones, and therefore to contribute to motoneuronal
            activity over a range of EMG and force levels. A similar
            distributed system of synaptic connections has also been
            observed for muscle spindle afferents (Mendell and Hennemann
            1971).

            There are a large number of possible sources of common input
            to spinal motoneurones, including segmental afferent inputs
            from muscle spindles and other peripheral receptors.
            However, several lines of evidence suggest that, at least
            for hand muscle motoneurones, much of the short-term
            synchrony has a corticospinal origin.

            Both refined histological and electro-anatomical
            observations on single corticomotoneuronal fibres in the
            monkey indicate significant divergence of the intraspinal
            collaterals of these fibres to impact on a large number of
            motoneurones. It is possible that, for some muscles acting
            about the distal joints of the limb, most or all of the
            motoneurones are engaged by synapses from each
            cortico-motoneuronal fibre which includes that muscle within
            its muscle field. The synaptic boutons that generate the
            cortico-motoneuronal synapses are small in size and only one
            or a few boutons are contributed to the motoneuron's
            dendritic surface by each cortico-motoneuronal collateral.
            The contribution to the synaptic excitation of a motoneurone
            that is made by any one cortico- motoneuronal fibre is
            small. However, because of the high degree of convergence of
            intraspinal collaterals from a large colony of
            cortico-motoneuronal neurones on to each motoneurone of a
            distally acting muscle, this synaptic influence could allow
            fine grading of the depolarization of the motoneurone,
            contribute to the selective activation of motor units during
            voluntary movement and be critically effective in the
            fractionation of muscle usage during different motor tasks.

            other literature (work on the extensor digitorum communis)

            Things we still don't fully know (that are probably beyond
            the scope of this project):

            \begin{enumerate}
            \def\labelenumi{\arabic{enumi}.}
            \tightlist
            \item
              How extensive is the CM projection?
            \item
              How many CM cells project to a given motoneuron or to a
              given muscle?
            \item
              how large are the postsynaptic CM effects in a given
              motoneuron? {[}Lemon 2008{]}
            \end{enumerate}

            Renshaw cells might play a role in isolating single MUs:

            It is concluded that Renshaw cells de-correlate discharge
            patterns of different motoneurones of the same pool by
            injecting uncorrelated signals into them. This decorrelation
            is an important prerequisite for distortion suppression of
            signal transmission in a multi-channel system, like that of
            stretch reflex, and for its linearization. {[}Adam 1978{]}

            Another report that people don't know how they're learning
            to isolate SMUs:

            Although subjects used auditory synchrony feedback, they
            were generally unable to evaluate their success in
            modulating the amount of MU synchrony; nor could they
            describe any conscious strategy used during the successful
            sessions. They concentrated on the

            occurrence of the synchrony pulses, rather than on the
            amount of EDC contraction or the concurrent force output.
            {[}Schmeid, 1993{]}

            Implications that CM cells are responsible for fast changes
            in MU dynamics:

            The present changes in short-term synchronization were
            detected in the course of the conditioning session and must
            reflect rapid alterations in the relative contributions of
            common inputs. One direct mechanism would involve a change
            in the proportion of descending monosynaptic input, such as
            the corticomotoneuronal cells that strongly affect EDC
            motoneurons (Phillips and Porter 1977; Fetz and Cheney
            1980). {[}Schmeid, 1993{]}

            Statement of unknowns about how dynamics are shifted at the
            MU level in terms of spinal circuitry:

            Indirect control of MU synchronization might also be
            mediated by supraspinal modulation of the Renshaw
            decorrelating action (Gelfand et al.~1963; Adam et
            al.~1978), or by enhanced activity of spindle afferents via
            selective activation of gamma-motoneurons (Rudomin 1989).
            These possibilities suggest further experiments to resolve
            the neural mechanisms by which humans could rapidly alter
            short-term synchrony of MUs and, by implication, control the
            proportion of last-order common inputs to motoneurons.
            {[}Schmeid, 1993{]}

            \hypertarget{motor-system-physiology}{%
            \section{Motor System
            Physiology}\label{motor-system-physiology}}

            Muscles are collections of fibers that contract when
            chemical gradients are produced at the neuromuscular
            junction by action potentials emanating from neurons in the
            ventral horn of the spinal cord.

            Electromyography is the detection of changes in chemical
            potential using electrodes. In my setup, we use a total 64
            monopolar surface electrodes and monopolar needle electrodes
            to record chemical potentials from muscles in the forearm
            and hand.

            \hypertarget{motor-units}{%
            \subsection{Motor Units}\label{motor-units}}

            \begin{quote}
            The motor unit, namely a motoneurone and the many muscle
            fibres singularly innervated by branches of the
            motoneurone's axon, is the quantal element underlying the
            transduction of neural commands driving the exquisite motor
            behaviours produced by the hand. (Fuglevand 2011)
            \end{quote}

            Experimental work characterizing motor unit properties in
            humans and other mammals has been consistent with regard to
            two findings(see Fuglevandet al.1993). First, twitch or
            tetanic forces of motor units that constitute a muscle vary
            over an extremely wide range, usually 100-fold or greater.
            And second, the frequency distribution of motor units
            according to force capacity is markedly skewed toward motor
            units that produce small forces, with few units that
            generate large forces. (Fuglevand 2011)

            One consequence of this organization is that fine resolution
            of force is an in-built control feature, such that when
            performing delicate motor tasks involving weak muscle
            contractions, subtle adjustments in force can be
            accomplished by drawing upon a large population of weak
            motor units. (\emph{this is logarthmic in the number of
            units recruited}) (Fuglevand 2011)

            \textbf{Motor unit number itself seems to play a critical
            role in determining precision of muscle force. When human
            subjects attempt to produce a constant force during
            isometric contractions, the force inadvertently fluctuates
            about the specified target level. Such force variability
            increases in roughly in proportion to the target force(Enoka
            et al.2003). Unexpectedly, this noise in force control is
            greater for hand muscles compared to more proximal muscles
            (Hamilton et al.2004). Furthermore,based on available
            estimates of motor unit numbers and computer simulation, a
            key factor underlying greater noisiness in hand muscles was
            relatively low numbers of motor units (Jones et al.2002;
            Hamilton et al.2004). In addition, augmented force
            variability in hand muscles maybe related to greater
            variability and common modulation in motor unit discharge
            rate compared to more proximal muscles (Negro et al.2009).
            Therefore, the widely held view that hand muscles are
            optimally designed for fine control may require
            reconsideration.} (Fuglevand 2011)

            While there are anatomical (Feinsteinet al.1955) and
            electrophysiological (McComaset al.1971; Bromberg,2007)
            means to estimate motor unit numbers in humans,both methods
            are susceptible to several sources of error.Perhaps the most
            reliable information at present available about relative
            numbers of motor units supplying different muscles comes
            from retrograde labelling of motoneurones in non-human
            primates. In such studies, intrinsic hand muscles have been
            shown to be innervated by ∼50--200 motoneurones, while more
            proximal muscles like biceps and triceps brachii are each
            supplied by more than 1000 motoneurones (Jenny \& Inukai,
            1983). (Fuglevand 2011)

            \hypertarget{kinematics}{%
            \subsection{Kinematics}\label{kinematics}}

            The human hand has approximately 20 degrees of freedom in
            its 19 joints, but the number of dominant modes of activity
            is much lower due to mechanical coupling.

            27 bones and 36 muscles (???)

            There are 31 muscles of different and often complex
            architecture involved in hand movement with 19 residing in
            the hand (the intrinsic hand muscles). The hand has 19
            articulations, 18 tendons crossing the wrist, and at least
            25 degrees of freedom. (Duinen \& Gandevia 2011)

            \hypertarget{forearm}{%
            \subsection{Forearm}\label{forearm}}

            the deep flexor attaches to the distal phalanx the
            superficial flexor attaches to the middle phalanx

            The main finger extensor, extensor digitorum (ED), when
            active generates torque about the elbow, wrist,
            metacarpalphalanegeal, proximal interphalangeal and distal
            interphalangeal joints simultaneously (Anet al.1981).
            Moreover, ED, likeits flexor counterparts, flexor digitorum
            superficialis and flexor digitorum profundus, gives rise to
            four distal tendons that insert into each of the
            fingers.Therefore, attempts to move a single finger in
            isolation require that other muscles be co-activated to
            counteract the unwanted actions produced by the agonist
            (Schieber, 1995; Valero-Cuevas, 2000). (Fuglevand 2011)

            Three major forearm muscles consist of multiple muscle
            bellies with tendons to each finger so that the muscles have
            four `compartments' (Duinen \& Gandevia 2011)

            Force and displacement interactions can occur within
            muscles, an issue of particular concern given that flexor
            digitorum profundus (FDP), flexor digitorum superficialis
            (FDS) and extensor digitorum (ED) are muscles with tendons
            to each of the fingers. These interactions may occur, for
            example, because a motor unit's territory is such that force
            is `injected' into more than one distal tendon. This sort of
            `lateral' force transmission exerted by individual muscle
            fibres and motor units can be significant in some animal
            preparations (e.g.~Street, 1983; see also Young et
            al.~2000). The topic of lateral force transmission is
            controversial but it appears that such an effect can even
            result in inter-muscle force transfer (e.g.~for review Patel
            \& Lieber, 1997; Huijing, 1999, 2009). This process has been
            most studied in the lower limb for gastrocnemius and soleus
            in animals (for details see Maas \& Sandercock, 2008) but
            some evidence exists for it in humans (e.g.~Bojsen-Moller et
            al.~2010). However, the unresolved issues are the magnitude
            of these effects and the conditions under which they are
            functionally significant. (Duinen \& Gandevia 2011)

            If hand muscles do not always behave as simple in-line
            motors, then the spread of their mechanical effects must
            depend on the links of force--length curve, viscoelastic
            properties, and the changes induced by muscle contractions.
            Such detailed biomechanical information is largely lacking.
            (Duinen \& Gandevia 2011)

            Spillover has been shown in experiments studying the
            `recruitment thresholds' (defined below) of motor units ac
            ting on other digits during single digit contractions
            (Kilbreath \& Gandevia, 1994; Butler et al.~2005; van Duinen
            et al.~2009). In these experiments, motor units were
            recorded from one (test) compartment of the respective
            muscles, while subjects were asked to contract the
            compartment of the other digits up to 50\% of their maximal
            force. When the subjects contracted these other digits (one
            by one), motor units of the test compartment were often
            recruited. The amount of force produced by the other digits
            at the time of recruitment of the motor unit of the test
            compartment is termed the recruitment threshold. The general
            finding for all three muscles was that, the closer the
            contracting compartment to the test finger, the more motor
            units were recruited. (Duinen \& Gandevia 2011)

            One has to ask whether this spillover is functional. Is the
            frequent recruitment of motor units ac ting on the little
            finger when we extend the thumb part of a fixed pattern of
            muscle activation, perhaps to balance forces around the
            wrist? (Duinen \& Gandevia 2011)

            Until recently, most studies looked at either flexion or
            extension, but when we compare the amount of enslavement in
            flexion and extension, the enslaved forces in extension are
            higher than in flexion, when recorded in the same apparatus.
            We hypothesise that the level of enslavement might depend on
            the amount of individual daily usage (for data on usage see
            Ingram et al.~2008). (Duinen \& Gandevia 2011)

            When multiple digits had to contract, the subjects were not
            able to reach their maximal force, thus showing a force
            `deficit'. These deficits may be comparable to those when
            trying to produce force with two hands or arms, a phenomenon
            known as the bilateral deficit (Gandevia, 2001).

            \hypertarget{hand}{%
            \subsection{Hand}\label{hand}}

            The intrinsic hand muscles can also be activated almost
            maximally (e.g.~Merton, 1954; Herbert \& Gandevia, 1996),
            but they are special in that they can be `controlled' at
            very low levels, even below the recruitment threshold for
            the earliest recruited units (Gandevia \& Rothwell, 1987).
            (Duinen \& Gandevia 2011)

            \hypertarget{thumb}{%
            \subsection{Thumb}\label{thumb}}

            The human thumb confers great scope for dexterity and its
            long length relative to the index gives it the highest
            `opposability index' among primates (Napier, 1972), while
            its rotated first metacarpal and unique carpometacarpal
            joint enhance its range of movement for grasping and
            manipulation (Wood-Jones, 1949). Furthermore, the thumb is
            moved by a muscle in the forearm, flexor pollicis longus
            (FPL), which provides the only way to flex its distal joint
            and is rudimentary in apes. (Hiske van Duinen and Simon C.
            Gandevia 2011)

            The presence of FPL in humans is associated with a high
            capacity to s ense thumb voluntary forces at remarkably low
            levels compared even to intrinsic hand muscles (muscles with
            their origin and insertion in the hand; Kilbreath \&
            Gandevia, 1993) and to detect length changes at its distal
            joint (Refshauge et al.~1998). (Hiske van Duinen and Simon
            C. Gandevia 2011)

            \hypertarget{muscle-spindles}{%
            \subsection{Muscle Spindles}\label{muscle-spindles}}

            Arm movements are sensed via distributed and individually
            ambiguous activity patterns of muscle spindles,which depend
            on relative joint configurations rather than the absolute
            hand position. Interpreting this high dimensional input
            (around 50 muscles for a human arm) of distributed
            information at the relevant behavioral level poses a
            challenging decoding problem for the central nervous system.
            Proprioceptive information from the receptors undergoes
            several processing steps before reaching somatosensory
            cortex (3,8) - from the spindles that synapse in Clarke's
            nucleus, to cuneate nucleus, thalamus (3,9), and finally to
            somatosensory cortex (S1). In cortex, a number of tuning
            properties have been observed, such as responsiveness to
            varied combinations of joints and muscle lengths (10,11),
            sensitivity to different loads and angles (12), and broad
            and unimodal tuning for movement direction during arm
            movements (11,13).The proprioceptive information in S1 is
            then hypothesized to serve as the basis of a wide variety of
            tasks, via its connections to motor cortex and higher
            somatosensory processing regions. (Sandbrink \& Mathis,
            2020)

            \hypertarget{experimental-methods}{%
            \section{Experimental Methods}\label{experimental-methods}}

            \hypertarget{prior-work}{%
            \subsection{Prior Work}\label{prior-work}}

            \hypertarget{high-dimensional-emg-feedback}{%
            \subsection{High-Dimensional EMG
            Feedback}\label{high-dimensional-emg-feedback}}

            The goal of the project's first phase is to develop a
            high-dimensional surface EMG recording rig to generate
            datasets with high signal-to-noise ratio and dense coverage
            over superficial muscles of the arm and hand. The first
            question of this phase is: what are the limitations of a
            closed-loop myocontrol experiment, and how can such
            constraints be avoided or leveraged? To answer this question
            we will develop a signal processing pipeline and diagnostics
            suite to identify constraints in the setup and aim to
            overcome, as much as possible, the limitations inherent in
            surface EMG recording such as muscle crosstalk and rigorous
            electrode placement{[}10{]}. The concept of the experimental
            setup is shown in Figure 1, where 64 monopolar electrodes
            are attached to a subject's arm and hand to record muscle
            activity. The arm and hand are kinematically constrained in
            a custom fixture and motor activity is recorded during
            isometric muscle contractions at levels less than 20\%
            maximum voluntary contraction to lessen the risk of
            involuntary co-contractions. The setup circumvents the limb
            biomechanics by mapping muscle output directly to virtual
            stimuli shown on a computer monitor. Additionally, our study
            focuses on low-force, isometric contractions to avoid
            complications due to artifacts in dynamic, high- force
            movements. We chose 64 channels in order to have at least
            two electrodes per muscle implicated in control of the hand
            in the event that we require differential recording. This
            choice limits our analysis to the motor pool level. If our
            questions require recording at the motor unit level, we will
            need to move to a higher channel count system. Literature in
            this field typically use a much lower number of channels. We
            believe that using 64 electrodes will help develop a more
            complete picture of the superficial muscle activity of the
            arm and hand across learning. A diagram of muscles relevant
            to thee control of the hand and wrist is shown in Figure 2
            on page 4. We are not aware of a rigorous study testing
            which muscles of the arm and hand can be accurately captured
            using surface EMG.

            The second question of this phase is: what is the manifold
            of activity in electrode space during natural hand use? To
            answer this question, we will record naturalistic activity
            by subjects completing a set protocol that covers the
            naturalistic space of electrode covariance. For comparison,
            we will record a dataset of naturalistic tasks using a
            separate, mobile setup with the same electrode placement
            pattern but without the isometric constraint. These datasets
            could be collected from a range subjects going throughout
            their daily tasks, or using a specific set of tasks in the
            laboratory such as handwriting and the use of various tools.
            Encouragingly, a recent review noted that ``Similarly to the
            breakthroughs in understanding vision that followed the
            quantification of statistics of natural scenes, a clear
            description of the statistics of natural tasks might
            revolutionize our understanding of the neural basis of
            high-level learning and decision- making''{[}18{]}. By
            analyzing the structure of these naturalistic datasets, we
            can compute the dimensionality of naturalistic movement as a
            subspace within our electrode space, similar to work done
            using joint angles of the hand{[}24, 22, 11{]}. From this
            work we know that while the hand has 29 joints and is
            controlled by 34 muscles, the dimensionality of natural hand
            movements is closer to 8 in joint angle dimension space
            based on principle components analysis. This analysis will
            also help us determine the biomechanical constraints on hand
            output dimensionality. We hypothesize that this will be
            higher than 8 and lower than 23, which gives us a large task
            space to work with for generating learning tasks. We aim to
            extend this prior work using learning algorithms that take
            into account time- varying dynamics of the signal in
            addition to common tools like components analysis and matrix
            factorization. This analysis will help generate an
            understanding of intersubject, intersession, and intertask
            variability. Both an analysis of dynamic correlations and a
            validation of dimensionality using EMG would be a novel
            contribution to the literature. We anticipate that
            quantifying electrode placement and calibrating across
            sessions will be a major challenge. We aim to develop a
            mechanical fixture for recording as well as alignment tools
            to aid in placing electrodes in precisely the same location
            each session. Properly separating variability due to
            electrode placement from behavioral and physiological
            variability will be paramount to establish repeatability in
            our results. Once we have collected a naturalistic activity
            dataset, we can begin to design bespoke feedback mappings
            and perturbations, as discussed in Section .

            \hypertarget{unsupervised-feature-extraction}{%
            \subsection{Unsupervised Feature
            Extraction}\label{unsupervised-feature-extraction}}

            We want to determine a redundant control space from data
            taken during natural activity. The difficulty with this is
            that such a natural activity manifold may display spatial
            (channel-wise) correlations that are possibly
            physiologically separable. Thus, there are two aims which
            must be addressed separately:

            \begin{enumerate}
            \def\labelenumi{\arabic{enumi}.}
            \tightlist
            \item
              Expore subjects' ability to decorrelate descending output
              to the muscles which have been shown to be correlated in a
              natural activity dataset.

              \begin{itemize}
              \tightlist
              \item
                Such a structured exploration might provide support for
                the hypothesis that ``synergies'' are flexible
                correlations between muscles driven by task demands
                rather than (or in addition to) physiological structure.
                This needs to be done incredibly carefully to escape
                criticism of hard-wired synergy enthusiasts.
              \item
                See \emph{de Rugy 2012} for a critique of OFC and
                hard-wired synergies
              \end{itemize}
            \item
              Use common correlated outputs to develop a family of
              BMI-type learning tasks as a proxy for a ``novel skill'',
              then track motor planning of this new skill to compare
              with motor planning algorithms.

              \begin{itemize}
              \tightlist
              \item
                We might be able to get \#1 for free by going after this
                goal if we're careful in the setup
              \item
                This is arguably a more impactful focus as it connects
                low-level motor hierarchy data (EMG) to high-level
                planning with a normative hypothesis.
              \end{itemize}
            \end{enumerate}

            Electrode data from a single trial of a single session is
            held in a data matrix \(X\) (n\_electrodes, n\_samples), and
            we wish to find a latent weight matrix \(W\) (n\_electrodes,
            n\_components) which reconstructs \(X\) by projecting latent
            trajectories \(H\) (n\_components, n\_samples) into
            electrode space:

            \[
            X = W\cdot{H}
            \]

            \(H\) is the activity of the latent processes, and \(W\) is
            there mixing matrix. The columns of \(W\) are the principal
            vectors spanning the latent subspace in electrode space. If
            we have new samples, we can project these new points onto
            this subspace:

            \[
            h_{new} = W^T\cdot{w_{new}}
            \]

            To justify this decomposition, we have to make some
            assumptions about the nature of the EMG signal, namely that
            the signal is linear instantaneous (each EMG sample can be
            instantly mapped to control space). The other assumption is
            that the basis \(W\) should be orthonormal, that the columns
            of \(W\) are orthogonal with unity norm. This ensures that
            the left inverse \(W^{-1}\) is equal to the transpose
            \(W^T\) such that:

            \[
            \begin{align}
            X &= W\cdot{H} \\
            W^{-1}\cdot{X} &= {H} \\
            W^{T}\cdot{X} &= {H}
            \end{align}
            \]

            See \emph{Muceli 2014} for use of the Moore-Penrose
            pseudoinverse in place of the transpose when the columns of
            \(W\) do not form an orthonormal basis. This would be the
            case for NMF. Is there a factorization that produces
            nonnegative, orthogonal coordinates? Or is the pseudoinverse
            okay? I will need to test this.

            Stated in an information theoretic way, we want to minimize
            the reconstruction loss \(\mathcal{L}\) for our derived
            encoder-decoder pair (\(E\),\(D\)). We're decoding high
            dimensional activity into its latent dimensions, and
            encoding back into the high dimensional space. :

            \[
            \min_{E,D}{\mathcal{L}\left[X - EDX\right]}
            \]

            This way, forget about orthonormality and solve for an
            encoder and decoder directly. That is, \(E\neq{D}\) is
            perfectly acceptable.

            Each row of \(D\) might be called a \textbf{spatial filter},
            a linear combination of electrode activities into a
            surrogate, hopefully more intuitive space.

            In general to find such a basis we must :

            \begin{itemize}
            \tightlist
            \item
              Extract ``natural activity manifold'' from freeform data
            \item
              Use features of this natural subspace to derive control
              mapping

              \begin{itemize}
              \tightlist
              \item
                Linear iid features:

                \begin{itemize}
                \tightlist
                \item
                  PCA
                \item
                  dPCA
                \item
                  NMF
                \item
                  ICA
                \end{itemize}
              \item
                Linear time-dependent features:

                \begin{itemize}
                \tightlist
                \item
                  SSA
                \item
                  LDS model / PGM
                \end{itemize}
              \item
                Nonlinear

                \begin{itemize}
                \tightlist
                \item
                  autoencoders
                \item
                  networks
                \end{itemize}
              \end{itemize}
            \end{itemize}

            The behaviors present in our calibration dataset are
            crucial, as they determine the spatial correlations used to
            generate the mapping. If only complex, multi-muscle
            movements are present in the calibration, it will be
            impossible to decode subtle movements involving few muscles.
            Additionally, because extraction is unsupervised, it will be
            impossible to know how to alter the control basis directions
            (if we wish to do so) such that they involve single muscles
            or the smallest sets of muscles.

            Ultimately, we want to find reproducible features in our
            data that are due to muscle coordination alone, rather than
            volitional movements. We want the lowest level covariance
            that reflects physiology rather than a task-level behavioral
            description (see \emph{Todorov, Ghahramani 2005} and
            \emph{Ingram, Wolpert 2009}). The idea is that if we collect
            data from enough tasks, we can extract the common modes of
            muscle activity. This is true only if we are sampling
            uniformly from the space of tasks. Otherwise one task, and
            therefore one coordination pattern, will be overrepresented.

            \hypertarget{task-formalization}{%
            \subsection{Task Formalization}\label{task-formalization}}

            In this task, the subject's first goal is to interact
            through an unknown visuomotor mapping and internalize this
            model. The second problem is to use this model to solve a
            control problem.

            \begin{enumerate}
            \def\labelenumi{\arabic{enumi}.}
            \tightlist
            \item
              System Identification -- learning a transition function
              \(p(y_t|x_t, u_t)\)

              \begin{itemize}
              \tightlist
              \item
                How do you learn the unknown observation model from
                data?
              \end{itemize}
            \item
              Policy Optimization

              \begin{itemize}
              \tightlist
              \item
                Once dynamics are learned (or at least stable?), how do
                we form a policy that is generalizable to new tasks
                under these dynamics?
              \item
                This is the control problem.
              \end{itemize}
            \end{enumerate}

            It's safe to assume that these processes are happening in
            parallel. Because we have complete and arbitrary control
            over the observation mapping, we can ask the subject to
            interact through a dynamic that is intuitive (informative
            prior) or unintuitive (uninformative or inhibitive prior).
            Each scenario, we hypothesize, will elicit different
            strategies for learning and control.

            The unknown mapping \(M\) from muscle to task space looks
            like the observation matrix \(H\) in the LQG problem:

            \[
            \begin{align*}
            y_t &= Hx_t + v_t\,\,\,(\mathrm{LQG}) \\
            y_t &= Mx_t + v_t. \,\,\,(\mathrm{experiment})
            \end{align*}
            \]

            The state dynamics in the task are of the form:

            \[
            \begin{align*}
            x_{t} &= Ax_{t-1} + Bu_{t-1} + w_{t-1} \,\,\,(\mathrm{LQG}) \\
            x_t &= Dx_{t-1} + Iu_{t-1} + w_{t-1} \,\,\,(\mathrm{experiment})
            \end{align*}
            \]

            where \(D\) is a diagonal decay matrix of with terms
            \(\mathrm{e}^{-\lambda}\) and \(I\) is the identity. The
            subject produces muscle contractions which add to the
            current latent (unobserved) state. In the absence of control
            signals, the state decays back to \(0\) in line with the
            physics of your arm returning to a passive state in the
            absence of muscle contractions. The terms \(w\) and \(v\)
            are gaussian noise vectors with distributions
            \(\mathcal{N}(0,Q)\) and \(\mathcal{N}(0,R)\). If we combine
            the transition and observation models:

            \[
            \begin{align*}
            y_t &= MDx_{t-1} + Mu_{t-1} + Mw_{t-1} + v_t \\
            &= A^\prime x_{t-1} + B^\prime u_{t-1} + Mw_{t-1} + v_t.
            \end{align*}
            \]

            We can think of this as the combined system identification
            problem where \(A^\prime=MD\) and \(B^\prime=M\) are unknown
            and must be estimated. The noise covariances of this altered
            system are now non-trivial, however. We could also assume
            that the transition dynamic \(D\) is known and that the
            identification problem is learning the mapping \(M\) only.
            This might not be a poor assumption since the exponential
            decay is meant to serve as an intuitive passive dynamic.

            In each trial of the task, a subject will have some internal
            representation of the observation dynamic \(M\) which may or
            may not be accurate. In order to make accurate predictions,
            \(M\) must be estimated.

            Learning linear dynamical systems from data is a hot topic
            of research, most of which seems to focus on learning in the
            context of complete state observation (\(M=I\), \(y=x\)).
            Algorithms to determine parameters of \(A\) and \(B\) are
            proposed (see Dean, Recht 2018).

            From LQG theory we know that the control law is a linear
            function of the state:

            \[
            \begin{align*}
            u_t = -L_tx_t
            \end{align*}
            \]

            and thus our combined system dynamic is:

            \[
            \begin{align*}
            y_t &= M(D-L_t)x_{t-1} + Mw_{t-1} + v_t.
            \end{align*}
            \]

            The noise covariance due to the observation Q is unchanged,
            but the new noise covariance for the latent process is now
            \(MRM^T\). This may make things difficult.

            \hypertarget{questions}{%
            \paragraph{Questions}\label{questions}}

            \begin{itemize}
            \tightlist
            \item
              In a behavioral experiment, how can you disentangle system
              identification/estimation and control? Is suboptimality
              due to one or the other?
            \item
              How does the observation mapping relate to the latent
              state covariance? The task state covariance?
            \item
              How do we formalize this into a probabilistic graphical
              model? Why would we?

              \begin{itemize}
              \tightlist
              \item
                Would this make it easier to reason about what the goals
                are?
              \item
                Would learning \(M\) become an inference problem?
              \item
                Would solving the control problem become an inference
                problem\ldots?
              \end{itemize}
            \item
              What noise assumptions can we make? Can we not make?

              \begin{itemize}
              \tightlist
              \item
                How can we incorporate signal-dependent noise?
              \end{itemize}
            \end{itemize}

            \hypertarget{model-based-reinforcement-learning}{%
            \subsubsection{Model-based Reinforcement
            Learning}\label{model-based-reinforcement-learning}}

            Since we only have an approximate model of the system
            dynamic, we could simply work towards an optimal policy
            directly using gradient derivative-free optimization methods
            in a model-free approach. Since we have good evidence that
            humans leverage internal models to make decisions (at least
            in a motor problem domain), we need to define an algorithm
            which uses past observations and controls to update our
            approximation for the system dynamic. Here is a very general
            algorithm:

            \begin{enumerate}
            \def\labelenumi{\arabic{enumi}.}
            \setcounter{enumi}{-1}
            \tightlist
            \item
              Define a base policy/controller and base system model
              (\(L_0\) and \(\hat{M}_0\))
            \item
              Collect samples (by interacting with the true environment
              \(M_{true}\)) using the current policy/controller (collect
              \(y_t,u_t,y_{t+1}\) triples using \(L_i\) for
              \(i \in \{0\dots N\}\)
            \item
              Use sample(s) / trajectories to update current system
              dynamical model \(\hat{M}_i\)
            \item
              Update current policy/controller \(L_i\) (using the system
              dynamics or using a direct policy method)
            \end{enumerate}

            If the true system dynamics were known, we could solve the
            Algebraic Riccati Equation with a backwards pass, and
            compute our controls in a forward pass. This general
            algorithm structure highlights how the (unknown) system
            identification and controller design are intertwined:
            identifying a system appropriately must rely on sampling and
            fitting regions of the state space pertinent to adequate
            control in terms of cost (Ross ICML 2012). Otherwise, our
            approximation to the true system dynamic will only produce a
            valid controller in regions we have previously explored. The
            question is how we can effectively (sample and time
            efficiently) utilize new state transitions we encounter
            either online as feedback or between trials to update our
            model and policy. That is, the number of trials and/or
            trajectories to use before updating either the system model
            and/or policy is an important parameter.

            In the LQG setting, this might be called ``adaptive LQG''.

            \hypertarget{questions-1}{%
            \paragraph{Questions}\label{questions-1}}

            \begin{itemize}
            \tightlist
            \item
              how does a subject sample the state space as to
              efficiently learn? do they sample optimally? how does
              controller/policy optimization proceed based on system
              identification?
            \item
              how does a human subject use error information from each
              trial and feedback from each time step to update their
              model and/or policy?

              \begin{itemize}
              \tightlist
              \item
                how does a subject balance policy updates with model
                updates?
              \end{itemize}
            \item
              On what scale (trials, timesteps) is the model altered?
              the policy?

              \begin{itemize}
              \tightlist
              \item
                Replanning at every timestep is a model predictive
                control algorithm
              \item
                What prediction can we make for ID/learning every trial?
              \end{itemize}
            \item
              how does a subject avoid ``distribution mismatch'' between
              their base policy and their optimal policy? How do they
              efficiently explore and use this new data to update their
              internal model?

              \begin{itemize}
              \tightlist
              \item
                what exploration strategy does a subject use to avoid
                mismatch?
              \item
                what
              \end{itemize}
            \item
              What is a subject's baseline/prior model?
              \(y_{t} = \hat{f}_0(x_t,u_t)\) or
              \(y_{t} \propto p_0(y_t|x_{t},u_t)\)
            \item
              What is the base policy / prior policy?
              \(u_t = \pi_0(\hat{x}_t)\)
            \item
              How do we think about learning a distribution over
              trajectories in control law space, or perhaps
              equivalently, in covariance/precision space?
            \item
              We might hypothesize that a subject will act as randomly
              as possible while minimizing cost, a maximum entropy
              solution that converges to an optimal controller?
              \(\mathcal{H}(p(u_t|x_t))\)
            \item
              How does a subject penalize changes to their controllers?
              Do they follow a KL-divergence type of measurement when
              improving their policy?
            \end{itemize}

            \hypertarget{data-analysis}{%
            \subsection{Data Analysis}\label{data-analysis}}

            \emph{Coming Soon}

            \hypertarget{analysis-of-apt-data}{%
            \subsection{Analysis of APT
            data}\label{analysis-of-apt-data}}

            \emph{Coming Soon}

            \hypertarget{theory}{%
            \section{Theory}\label{theory}}

            \hypertarget{error-based-learning-1-day}{%
            \subsection{Error-based Learning (1
            day)}\label{error-based-learning-1-day}}

            Error-based sdaption and state-space models have a great
            amount of precedent in the sensorimotor learning literature.
            We will summarize these models briefly and discuss our
            willingness to depart from them.

            Current models of motor learning

            x' = Ax + Bu

            This model describes\ldots{}

            The downsides of this model are that it descibes a small
            aspect of our data.

            \hypertarget{optimal-feedback-control}{%
            \subsection{Optimal Feedback
            Control}\label{optimal-feedback-control}}

            The control setup writes a cost, environment has some
            dynamics.

            What is changing in this scenario? What is being learned?
            What information is used to do this learning?

            Which model variables correspond to muscles? Movements? What
            does the resultant feedback controller compute? How does
            this relate to cognition?

            Is LQR (as it's claimed to be) a reasonable model for
            feedback control and error reduction + variability
            prediction for dimensionality reduction-based motor
            interface (task reads out from D muscles, find modes of that
            data; do PCA to get K \textless{} D dimensions, controller
            only responds to motion in those K directions)---does
            behavior + motor activity follow LQR? this question has
            already been asked, but it hasn't been asked for this kind
            of high-to-low dim mapping. It's been asked in tasks where
            muscles haven't been directly in control (Bolero 2009).
            Todorov: do a task, look at muscle signal. Muscles that
            aren't necessary for task have higher variability b/c
            they're not being optimized for task (but does't introduce
            perturbations). Also see Loeb (2012) for a negative result
            saying that muscle coordination is habitual rather than
            optimal, but it has issues (low \# muscles). Can we
            replicate previous reaching optimality results in our
            set-up? What's unique about our set-up is the
            PCA/dimensionality reduction in muscle activity space. This
            is important because you can create arbitrary muscle-cursor
            mappings, so you have to learn a new skill/mapping. This is
            different than perturbing a fundamental movement and forcing
            adaptation, which is what has been previously done. For our
            task, the participants actually have to learn a new
            task/mapping, rather than just do what they already know and
            be robust to perturbations. We test the LQR hypothesis once
            they've learned the task, because LQR isn't a learning
            theory, it's a theory about optimal control. We can see if,
            once people learn a new skill, their behavior is optimal wrt
            LQR theory. If we establish this, then we can think about
            how this LQR model is actually learned (enter RL).

            This model is lacking in \ldots{}

            \hypertarget{intuitive-example-of-the-ofc-framework}{%
            \subsubsection{Intuitive Example of the OFC
            framework}\label{intuitive-example-of-the-ofc-framework}}

            Here we see a feedback controller with three muscles such
            that we can plot the muscle activation trajectory.

            This is the feedback controller K, we can understand it's
            action by plotting

            PLOT OF SIMPLE EXAMPLE

            PLOT OF APT DATA

            \hypertarget{composition-and-selection}{%
            \subsection{Composition and
            Selection}\label{composition-and-selection}}

            Here we'll review and discuss models of action selection and
            policy composition as a means of theorizing about how
            subjects learn novel skills.

            In a sense, we're setting up several different directions
            for our understanding of composition and action selection
            which can be experimentally tested.

            We have a direct selection algorithm, composition through
            policy addition, and composition through policy
            multiplication.

            \hypertarget{kl-control-composition-1-day}{%
            \subsubsection{KL-control Composition (1
            day)}\label{kl-control-composition-1-day}}

            This setup is particular subset of OFC problems.

            Dynamics Cost

            Composable policies

            PLOT OF INTUITIVE EXAMPLE

            \hypertarget{generalized-policy-selection-1-day}{%
            \subsubsection{Generalized Policy Selection (1
            day)}\label{generalized-policy-selection-1-day}}

            This is in the MDP case

            Learning happens in several ways-- reward regression,
            Q-learning

            What are rewards? What are tasks? What are actions?

            \hypertarget{next-steps}{%
            \section{Next Steps}\label{next-steps}}

            \hypertarget{gpi-for-ofc-1-day}{%
            \subsection{GPI for OFC (1 day)}\label{gpi-for-ofc-1-day}}

            Write up written notes on this

            \begin{enumerate}
            \def\labelenumi{\arabic{enumi}.}
            \setcounter{enumi}{1}
            \tightlist
            \item
              Is GPI with LQRs / LQR-RL a good model for motor learning?
              Define a model and see if it recapitulates known motor
              learning phenomena on existing experiments + accounts for
              things that previous models don't. (Similar in spirit to
              Geerts et al.~(2020)). Can this model track the
              higher-order statistics of trajectories during motor
              learning?
            \end{enumerate}

            \hypertarget{transfer-tasks-in-virtual-environment}{%
            \subsection{Transfer Tasks in Virtual
            Environment}\label{transfer-tasks-in-virtual-environment}}

            \hypertarget{preliminary-data}{%
            \subsubsection{Preliminary Data}\label{preliminary-data}}

            Analyze prelim data from Andy

            Our preliminary data confirms the working principle of the
            setup and highlights the next steps for producing a quality
            dataset.

            We must - address the noise concerns in the data - formalize
            specific task designs which link with our theoretical
            interests

            \hypertarget{bibliography}{%
            \subsection*{Bibliography}\label{bibliography}}
            \addcontentsline{toc}{subsection}{Bibliography}

            \hypertarget{refs}{}
            \begin{CSLReferences}{0}{0}
            \leavevmode\hypertarget{ref-McNamee2019}{}%
            \CSLLeftMargin{1. }
            \CSLRightInline{McNamee, D. \& Wolpert, D. M. Internal
            {Models} in {Biological Control}. \emph{Annual Review of
            Control, Robotics, and Autonomous Systems} \textbf{2},
            339--364 (2019).}

            \leavevmode\hypertarget{ref-DAvella2003}{}%
            \CSLLeftMargin{2. }
            \CSLRightInline{D'Avella, A., Saltiel, P. \& Bizzi, E.
            Combinations of muscle synergies in the construction of a
            natural motor behavior. \emph{Nature Neuroscience}
            \textbf{6}, 300--308 (2003).}

            \leavevmode\hypertarget{ref-DeRugy2012}{}%
            \CSLLeftMargin{3. }
            \CSLRightInline{de Rugy, A., Loeb, G. E. \& Carroll, T. J.
            Muscle {Coordination Is Habitual Rather} than {Optimal}.
            \emph{Journal of Neuroscience} \textbf{32}, 7384--7391
            (2012).}

            \leavevmode\hypertarget{ref-Takei2017}{}%
            \CSLLeftMargin{4. }
            \CSLRightInline{Takei, T., Confais, J., Tomatsu, S., Oya, T.
            \& Seki, K. Neural basis for hand muscle synergies in the
            primate spinal cord. \emph{Proceedings of the National
            Academy of Sciences} \textbf{114}, 8643--8648 (2017).}

            \leavevmode\hypertarget{ref-Rathelot2009}{}%
            \CSLLeftMargin{5. }
            \CSLRightInline{Rathelot, J.-A. \& Strick, P. L.
            Subdivisions of primary motor cortex based on
            cortico-motoneuronal cells. \emph{Proceedings of the
            National Academy of Sciences} \textbf{106}, 918--923
            (2009).}

            \end{CSLReferences}
                      </div>
      </div>


  </body>
</html>