\title{Sensorimotor Learning\\in Virtual Environments}
\author{Spencer Ryan Wilson}
\department{Sainsbury Wellcome Centre for Neural Circuits and Behaviour}

\maketitle

\makedeclaration

\begin{abstract} % 300 word limit

The computational underpinnings of human motor learning remain an exciting frontier at the intersection of neuroscience, statistics, and engineering. This work sits at that intersection, exploring the computational principles and strategies employed by the brain during acquisition and refinement of a dexterous motor skill. We begin by reviewing relevant motor physiology research and illustrate how hallmarks of human motor skill learning are evident in the architecture of the motor system. We provide an breakdown of the experimental design, hardware, and software used in this work for the benefit of our future colleagues. Our analysis begins with an overview of task performance data across subjects to confirm the integrity of our experimental design. We then explore subjects' electromyography data manifolds, providing a foundation for testable hypotheses about the evolution of muscle activations across learning. Next, we employ mixture models to extract statistics from our EMG data and refine hypotheses on the structure of subject variability. We then test hypotheses pertaining to the structure of task-relevant and task-irrelevant variability subspaces. We find that over the course of learning subjects identify and hone task-specific solutions with decreasing task-irrelevant variability, suggesting that subjects leverage a ``model-free'' learning strategy. We conclude with a discussion of possible future directions for the research program developed here. We argue that using high-dimensional EMG data to inform models of human learning is a powerful approach towards understanding our remarkable ability to quickly adapt in new and arbitrary contexts. Finally, we issue a call-to-action towards deeper interdisciplinary collaborations bridging human motor experiments with computational analysis in an effort to reverse engineer the complexity of the intelligent movement machine.

\end{abstract}

\begin{impactstatement}
% \begin{quote}
% The statement should describe, in no more than 500 words, how the expertise, knowledge, analysis,
% discovery or insight presented in your thesis could be put to a beneficial use. Consider benefits both
% inside and outside academia and the ways in which these benefits could be brought about.

% The benefits inside academia could be to the discipline and future scholarship, research methods or
% methodology, the curriculum; they might be within your research area and potentially within other
% research areas.

% The benefits outside academia could occur to commercial activity, social enterprise, professional
% practice, clinical use, public health, public policy design, public service delivery, laws, public
% discourse, culture, the quality of the environment or quality of life.

% The impact could occur locally, regionally, nationally or internationally, to individuals, communities or
% organisations and could be immediate or occur incrementally, in the context of a broader field of
% research, over many years, decades or longer.

% Impact could be brought about through disseminating outputs (either in scholarly journals or
% elsewhere such as specialist or mainstream media), education, public engagement, translational
% research, commercial and social enterprise activity, engaging with public policy makers and public
% service delivery practitioners, influencing ministers, collaborating with academics and non-academics
% etc.

% Further information including a searchable list of hundreds of examples of UCL impact outside of
% academia please see \url{https://www.ucl.ac.uk/impact/}. For thousands more examples, please see
% \url{http://results.ref.ac.uk/Results/SelectUoa}.
% \end{quote}
The framework used in this thesis, combining experimental and computational methods, for understanding the principles underlying human dexterous motor skill learning holds the potential for wide-ranging impact both within and outside of academia.

Within academia, this work lays the groundwork for an emerging interdisciplinary field interleaving concepts from motor neuroscience, multivariate statistics, and engineering. The experimental paradigm and analysis techniques presented herein provide a blueprint for future scholarship focused on dissecting the computational mechanisms of human, as well as general biological, motor learning. Translating empirical findings into testable hypotheses about the motor system's adaptive strategies enables development of new theoretical models capturing the magnificent complexity of human motor behavior. The curriculum in fields like neuroscience, bioengineering, and robotics could be enriched by incorporating the interdisciplinary perspectives championed here.

Outside of academia, translating the biological learning mechanisms uncovered in this work into computational algorithms could have impact in a range of applications. In robotics and prosthetics, replicating the neural strategies for rapidly adapting and generalizing motor skills could produce more intelligent, life-like systems capable of seamless skill acquisition in real-world environments. Similar adaptive algorithms could enhance human-machine interfaces by allowing more natural, intuitive control and physical interaction modalities. Technologies in rehabilitation and human performance optimization could leverage biological learning principles to create powerful tools maximizing the brain's ability to acquire and refine complex motor patterns.
\end{impactstatement}

\begin{acknowledgements}
    
    This has been a journey, and as such there are more than a few people to thank for their support and kindness. In somewhat chronological order, thanks to: Kelly Clancy for answering my email, Tom Mrsic-Flogel for his patience, Adam Kampff for bootcamp, all of my PhD cohort $\pm1$ year but especially Peter Vincent \& Philip Shamash, Andy Murray and Maneesh Sahani for their patience, Patrick Kaifosh and the Ctrl-Labs team, the FabLab crew (thanks guys!), Georgia Rosalie Lotter for her support and kindness, Klara Olofsdotter for her smoothing of wrinkles, Tiago Branco (a genuine legend), Michele Tarawneh (habibi!), Talfan Evans (I owe you brother), Kate Feenstra for her inspiration (Old French, divine guidance; Latin, breathe into). And I thank those whose love lives outside of time: my Mom (hi Mom!) and my Dad and my sister and my brother, I love you!

\end{acknowledgements}

\setcounter{tocdepth}{2}
\cleardoublepage\tableofcontents

\cleardoublepage\listoffigures

% \cleardoublepage\listoftables