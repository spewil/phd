\documentclass[../main.tex]{subfiles}
\graphicspath{{\subfix{../../images/}}}
\begin{document}

\chapter{Introduction \& Aims}\label{chap:intro}

\bigskip
\begin{quote}
  \emph{Movement is nothing but the quality of our being.}
  
  \raggedleft{--- Sunryu Suzuki, \emph{Zen Mind, Beginner's Mind} (1970)}
\end{quote}

\bigskip
\begin{quote}
  \emph{What I cannot create, I do not understand.}
  
  \raggedleft{--- Richard Feynman's final blackboard (1988)}
\end{quote}

\cleardoublepage%








\section{Inspiration from the Past}

In 1939, a Bell Labs researcher named Homer Dudley engineered a demonstration device he called the ``Voice Operation Demonstrator'' or ``Voder'' for the New York World's Fair\cite{dudleyAutomaticSynthesisSpeech1939}. Dudley explains:
%
\begin{quote}
  In the Fair demonstration, \ldots [the operator replies to an announcer's] questions by forming sounds on the voder and connecting them into words and sentences. She does this by manipulating fourteen keys with her fingers, a bar with her left wrist and a pedal with her right foot. \textbf{This requires considerable skill by the operator}.
\end{quote}
%
\begin{figure}
  \centering
  \begin{minipage}{0.59\textwidth}
    \includegraphics[width=\textwidth]{introduction/voder_schematic.jpg}
    \subcaption{}
  \end{minipage}
  \begin{minipage}{0.4\textwidth}
    \includegraphics[width=\textwidth]{introduction/voder.png}
    \subcaption{}
  \end{minipage}
  \caption[The Voder at the 1939 World's Fair]{The Voder project was a groundbreaking synthesis device to electronically produce human-like speech. The operator used 14 keys, a wrist bar, and a foot pedal in real time to manipulate electronic circuitry which generated speech-sound waveforms: vowels, consonants, and inflections. To use the device expertly required months of training. (a) Schematic of the Voder from the original paper. (b) Photo of an operator using the switchboard of the device.}\label{fig:voder}
\end{figure}
%
While Dudley's work served as an engineering investigation to reproduce speech sounds mechanically in line with Bell Telephone's goals of transmitting such sounds along their wires, he effectively invented a speech prosthetic controlled by the muscles of the body rather than the muscles of the tongue and larynx\endnote{For the armchair historians, see Wolfgang von Kempelen's (inventor of the ``mechanical Turk'' chess-playing automaton) ``speaking machine'' from 1780 as an earlier example of such experiments. Note that Von Kempelen spent twenty-odd years developing his speaking machine, whereas this thesis required a mere six!}. By reconfiguring the inputs from the muscles usually used for speech production to the many muscles controlling the hand, arm, and feet of operator of the the Voder device, Dudley created an experimental environment for the operator, now a subject, to solve a redundant motor task. To achieve success in their task, the operator had to develop ``considerable skill'', as Dudley pointed out. It is known anecdotally that there were only a few Voder operators, and they each trained for months to produce intelligible answers to unrehearsed questions using the device.

The experimental environment that Dudley created in 1939 is in exactly the same spirit as what we attempted to achieve with this work. The principle aim of this thesis is \textit{to construct an experimental platform on which a suite of motor skill learning experiments can be devised and recorded.} By developing and sharing this platform, we hope to advance our understanding of trial-to-trial motor skill learning through the combination of theory and experiment, beginning a process of data generation to yield insights into how human beings are able to solve complex motor tasks.

% Variety in experiments -- not just repeated movements
% In a talk by Emo Todorov, the renowned motor control theorist nots that we know how to design and interpret experiments that involve many repetitions of the same movement however there is limited role for online optimization in that context. instead \textbf{we need experiments where subjects are required to come up with new movements all the time. how can we get experimenters to do such experiments? show cool movies of robots doing cool things,and hopefully get the experimenters excited.} (todorov online optimization slides) The long term goal of the research direction suggested here is to develop tasks which ask subjects to produce a variety of movements in response to a variety of goals and perturbations. This will allow us to study the computations that underpin the remarkable human ability to solve the motor problems they constantly face. This stands in contrast to studies with many repetitions of the same movements. That said, we wish to validate our experimental setup on classical tasks as a stepping stone to tasks with greater variety.

% Maybe use big motor learning review to show we know what we're after...?
% Krakauer et al.'s motor learning review categorization of motor learning places prior work into the following classes: - Adaptation - Sequence Learning - De Novo Learning - Motor Acuity - Expertise. Align our interest and work into these categories and discuss how we're thinking about this...

% Our task sets up the case when these models are totally wrong-- not just a rotation (somewhere in the stack), but completely new contingencies entirely?

% procedural knowledge -- unable to express
% declarative knowledge -- able to articulate
% Reading from Kandel \cite{kandelPrinciplesNeuralScience2013}
% we're interested more in how subjects acquire procedural knowledge through experience
% Much of our movement, even what we term \textit{voluntary movement} is subconscious. Conscious thinking can actually impair performance, which is relatable if you have ever taken up a new sport. It is common to feel ``stiff'' and clumsy when, say, trying to hold in your head every aspect of perfecting your golf swing as an amateur. Coaches talk about ``letting your body think for you'' when it comes to deliberate practice. 



\section{The Movement Paradox}

Humans have sought to understand the nature of intelligence, that elusive quality that allows us to perceive, reason, and act upon the world around us, from our introspective beginnings. For all our intellectual ambitions, we have yet to replicate one of the most fundamental aspects of intelligence: the ability to move with the dexterity. Robotic motion excels at repetitive tasks in controlled environments, but tends to falter when faced with the ever-changing complexities of the real world. 

Hans Moravec's eponymous paradox states that it is easier to generate artificially intelligent performance on tasks we think of as intellectually challenging, such as chess, than to provide a machine with faculties we take for granted, such as movement. Moravec's Paradox, for example, encourages us not to look past the complex computations generated by the human motor system. Following Moravec, this work focuses on what is arguably the most advanced control apparatus in the known universe: the human hand. 

The human hand is the pinnacle of evolutionary advancement, a marvel of evolutionary engineering. Muscles, tendons, and bones work in elegant harmony, governed by a neural processor which seamlessly integrates perception, cognition, and action. By working to disentangle the algorithms employed by the human brain to acquire and refine motor skills, we hope to inspire novel machine learning architectures that replicate and surpass our own capabilities.

A recent review presented a compelling call to action to pursue research programs along exactly these lines:
%
\begin{quote}
  \textit{The processes by which biological control solutions spanning large and continuous state spaces are constructed remain relatively unexplored. Future investigations may need to embed rich dynamical interactions between object dynamics and task goals in novel and complex movements\cite{McNamee2019}.}
\end{quote}
%
This project is interested in observing an exceedingly complex, high-dimensional system which, faced with a new challenge, which optimizes itself towards a goal.






\section{The Approach}

To explore dexterous movement, we will leverage recordings of muscles controlling the hand as a readout of flexible motor behavior. This is a step beyond recording hand kinematics, as electromyography provides a direct, physiological output of the nervous system. Surface electromyography (EMG) recordings taken from the forearms controlling subjects' dominant hands allows us to track the sequential selection of muscle activations during both skill acquisition and subsequent performance of that skill to achieve desired goal.

As we are interested in subjects' abilities to acquire new skills, our goal is to design tasks which require subjects to use available, but uncommon, motor activations. We then track the selection and execution of these activations during those tasks for further study. Ideally, subjects will have no prior knowledge related to the task, requiring a balance of exploration and exploitation in order to succeed. Subjects will experience constraints due to the task contingency, as well as the limits of the underlying manifold of their motor activity. Our aim is to follow the progress of learning to gain an insight into how subjects cope with their new context.

Using data from our experimental setup, we aim to explore specifically how the structure of muscle activation variability evolves across skill acquisition, and how the motor system constructs skilled movement through the composition of component muscle activations. By carefully titrating task difficulty for ``slow learning'' over many trials, we expect to finely resolve the progression of how redundancies are harnessed and strategies explored as performance emerges. By prompting subjects to elicit uncommon muscle activation patterns in service of the task goals, we aim to increase the ecological validity of our experimental context beyond typical motor behaviors in order to shed light on the flexibility of the motor system. We believe that our paradigm provides a powerful platform to directly observe the neural processes involved as the redundant, dexterous motor system constructs a novel skill \textit{de novo}.














\section{Outline}

In \Cref{chap:background}, we review selected works of motor physiology research relevant to dexterous motor computations, and present a view of motor skill learning based on what is known about the structure of the motor system. In \Cref{chap:methods}, we provide a detailed explanation of our experimental design as well as the hardware and software underpinning the experimental apparatus. In \Cref{chap:performance}, we support the choices made in our experimental design by presenting data pertaining to task performance across subjects. In \Cref{chap:data_manifold}, we explore subjects' EMG manifolds to provide a foundation for testable hypotheses pertaining to the evolution of subjects' muscle activations across learning. In \Cref{chap:gmms}, we present mixture models to extract statistics from subjects' activations and refine our hypotheses for the structure of subject variability. In \Cref{chap:nullspace}, we test hypotheses pertaining to the structure of variability in terms of the task-relevant and task-irrelevant activity subspaces. In \Cref{chap:conclusion}, we discuss possible future directions for the research program we have presented in this work.

% In \Cref{chap:reinforce}, we compare what we have learned about subject variability to a computational model based on the theory of reinforcement learning.

\cleardoublepage\printendnotes%
\ifSubfilesClassLoaded{%
    \newpage%
    \bibliography{../bib/bibliography}%
}{}%
\end{document}