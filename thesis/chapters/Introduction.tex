\documentclass[../main.tex]{subfiles}
\graphicspath{{\subfix{../../images/}}}
\begin{document}

\chapter{Introduction \& Aims}\label{chap:intro}

\bigskip
\begin{quote}
  \emph{Movement is nothing but the quality of our being.}
  
  \raggedleft{--- Sunryu Suzuki, \emph{Zen Mind, Beginner's Mind}}
\end{quote}

\cleardoublepage%




The purpose of this thesis is to construct a high-dimensional electromyography recording setup as a platform on which a suite of motor control and learning experiments can be explored. With this novel setup, we hope to advance our understanding of trial-to-trial motor learning through the combination of experiment and theory.



Inspo!

Bell Labs Vocoder paper\endnote{https://archive.org/details/bellsystemtechni19amerrich/page/501/mode/1up?view=theater}


The long term goal of the research direction suggested here is to develop tasks which ask subjects to produce a variety of movements in response to a variety of goals and perturbations. This will allow us to study the computations that humans use in everyday tasks to solve the motor problems they face. This stands in contrast to many repetitions of the same movements. However, we wish to validate our experimental setup on classical tasks as a stepping stone to tasks with greater variety.



"we know how to design and interpret experiments that involve many repetitions of the same movement however there is limited role for online optimization in that context. instead we need experiments where subjects are required to come up with new movements all the time. how can we get experimenters to do such experiments? show cool movies of robots doing cool things,and hopefully get the experimenters excited." (todorov online optimization slides)



Inspiration from Basmajian
\begin{figure}
  \centering
  \includegraphics[width=0.2\textwidth]{introduction/basmajian_63.png}
  \caption{???}
  \label{fig:basmajian_63}
\end{figure}


Over the last few decades, there has been considerable amount of work done to untangle the abilities of the motor system to flexibly control the body including through optimal control theory\cite{Todorov2004}, reinforcement learning in continuous action space\cite{koberReinforcementLearningRobotics2013}, and detailed physiological studies\cite{sauerbreiCorticalPatternGeneration2019}. However, as the quote above suggests, a holistic understanding of the computations underlying the construction of skilled movement remains an exciting direction for research. Our aim is to progress understanding of skilled movement by studying the solutions produced by human subjects to motor tasks in dynamically rich, yet controlled, virtual environments. Our goal is to reverse-engineer the ability to acquire and perform novel motor skills.

Humans produce a great variety of movements every day, often without conscious thought. For example, movements like bringing a cup of coffee to our lips for a sip are generally out of reach for state-of-the-art robotic systems. We claim that this ``motor gap'' between biological and artificial motor systems is due to a lack of \emph{dexterity}. Soviet neuroscientist Nikolai Bernstein defined dexterity as the ability to ``find a motor solution in any situation and in any condition.''\cite{Bernstein1967} The crux of this definition is the flexibility of such solutions. This flexibility, or robustness\endnote{Kitano defines robustness as ``the maintenance of   specific functionalities of the system against perturbations, and it   often requires the system to change its mode of operation in a   flexible way.'' He claims that robustness requires control,   alternative mechanisms, modularity and decoupling between high and low   level variability.}\cite{kitanoBiologicalRobustness2004}, is the ability to optimize internal parameters in response to external perturbations and adapt to new information to achieve the goals of an ongoing plan.

To explore dexterous movement, we will leverage recordings of muscles controlling the hand as a readout of flexible motor behavior. This is a step beyond recording hand kinematics as electromyography provides a physiological output of the nervous system. Surface electromyography recordings taken from the forearms controlling subjects' dominant hands allows us to track the sequential selection of muscle activations during both skill acquisition and subsequent performance of that skill to achieve desired goal. As we are interested in subjects' abilities to acquire new skills, we design tasks that require subjects to use available, but uncommon, motor activations. We then track the selection and execution of these activation during virtual tasks. Preliminary work in this direction is described in \cref{sec:experiment}.

Using data from our experimental setup, we wish to understand both how the structure of muscle activation variability evolves during skill acquisition and how the motor system constructs skilled movement through the composition of component muscle activations. To begin, we review a  sampling of current motor physiology research relevant to dexterous motor computations in \cref{sec:physiology}. In \cref{sec:experiment}, we cover our prototype hardware and experiments. With inspiration from physiology and our experiments, we hope to make progress in modeling sensorimotor control and learning in our experimental setup. We cover preliminary work in this direction in \cref{sec:theory}, and discuss possible future directions in \cref{sec:next_steps}.

Why can't robots move like humans? What is special about human movement?
movement is a really hard problem, we want to understand why
Hans Moravec's eponymous paradox states that it is easier to generate artificially intelligent performance on tasks we think of as intellectually challenging, such as chess, than to provide a machine with faculties we take for granted, such as movement. Moravec's Paradox, for example, encourages us not to look past the complex computations generated by the human motor system. Following Moravec, this work focuses on what is arguably the most advanced control apparatus in the known universe: the human movement machine.
humans have extraordinary motoric ability, we want to understand why
Robustness, Flexibility, Generalization, Composition
Why is this the most interesting problem? 
The interesting problem here is coordination of a redundant system to produce dexterous solutions-- we want to solve the redundancy problem and produce solutions that are robust to external perturbations and sensitive to new information
While a robot may be able to move a cup of coffee to a precise location in space, its solution is often found to be brittle in a new context, or unable to generalize to the movement of new objects. We define a skill as a behavior that involves dexterity in Bernstein's sense. The use of a tool such as a screwdriver is an example of a motor skill. We define a task as the production of skilled movement in a particular context. Driving a screw in a particular posture using a particular screwdriver is an example of a task. These concepts will be further formalized in later chapters. 
Physiology is special for human hands so we chose them as a testbed

Human movement is ultimately the result of the activation and contraction of muscle fibers, and movements lie on a spectrum between reflexive and volitional. The supramuscular circuitry which determines the degree of volition we ascribe to movement, where volitional movement relies on supraspinal (though not necessarily conscious) processes. The human hand is a unique evolutionary invention that underlies our ability to perform various skills in a range of tasks-- movements that are decidedly volitional^[It could be argued that the hand is in fact a crucial aspect of humanness. It is thought that the human cerebellar and neocortices evolved reciprocally to expand and support the computational burden of increasingly complex motor tasks such as tool-making and language production[REF?]. The hand is the pinnacle of dexterity and, as such, it is a fruitful testbed for studying the computations and circuitry that drive dexterous movement. A detailed physiological review of the hand and it's relation to skilled movement is described in {+@sec:physiology}.

Setting up a specific experiment will help us to track muscle-level changes
we think theorizing with control and learning models will help us
what exactly from the theory world will help us? why is control/RL relevant?
what are we missing in the neuro / behavior lit that we need to borrow?

BE MORE SPECIFIC HERE

Admit that we need to collect and analyze data in an exploratory manner to then inspire hypotheses that can be modeled. We want these hypotheses to be inspired by theoretical work in control and learning theory
how are value computations connected to action and policy selec
To do so, we with the conceptual language of the experimental sensorimotor learning community together with the language of the control theory and reinforcement learning community, as each of these communities shares a common goal of understanding the computation underlying the production of skilled movement.


<!-- Why can't robots move like humans? What is special about human movement? -->
<!-- movement is a really hard problem, we want to understand why -->

The purpose of this thesis is to construct a high-dimensional electromyography recording setup as a platform on which a suite of motor control and learning experiments can be explored. With this novel setup, we hope to advance our understanding of trial-to-trial motor learning through the combination of experiment and theory.

<!-- Hans Moravec's eponymous paradox states that it is easier to generate artificially intelligent performance on tasks we think of as intellectually challenging, such as chess, than to provide a machine with faculties we take for granted, such as movement. Moravec's Paradox, for example, encourages us not to look past the complex computations generated by the human motor system. Following Moravec, this work focuses on what is arguably the most advanced control apparatus in the known universe: the human movement machine. -->


A recent review provides a clear call to action for work this direction:

> The processes by which biological control solutions spanning large and continuous state spaces are constructed remain relatively unexplored. Future investigations may need to embed rich dynamical interactions between object dynamics and task goals in novel and complex movements [@McNamee2019].

Over the last few decades, there has been considerable amount of work done to untangle the abilities of the motor system to flexibly control the body including through optimal control theory[@Todorov2004], reinforcement learning in continuous action space[@koberReinforcementLearningRobotics2013], and detailed physiological studies[@sauerbreiCorticalPatternGeneration2019]. However, as the quote above suggests, a holistic understanding of the computations underlying the construction of skilled movement remains an exciting direction for research. Our aim is to progress understanding of skilled movement by studying the solutions produced by human subjects to motor tasks in dynamically rich, yet controlled, virtual environments. Our goal is to reverse-engineer the ability to acquire and perform novel motor skills.

<!-- humans have extraordinary motoric ability, we want to understand why -->
<!-- Robustness, Flexibility, Generalization, Composition -->
<!-- Why is this the most interesting problem?  -->

<!-- The interesting problem here is coordination of a redundant system to produce dexterous solutions-- we want to solve the redundancy problem and produce solutions that are robust to external perturbations and sensitive to new information -->

Humans produce a great variety of movements every day, often without conscious thought. For example, movements like bringing a cup of coffee to our lips for a sip are generally out of reach for state-of-the-art robotic systems. We claim that this "motor gap" between biological and artificial motor systems is due to a lack of *dexterity*. Soviet neuroscientist Nikolai Bernstein defined dexterity as the ability to "find a motor solution in any situation and in any condition."\cite{Bernstein1967} The crux of this definition is the flexibility of such solutions. This flexibility, or robustness\endnote{Kitano defines robustness as "the maintenance of specific functionalities of the system against perturbations, and it often requires the system to change its mode of operation in a flexible way." He claims that robustness requires control, alternative mechanisms, modularity and decoupling between high and low level variability.\Cite{kitanoBiologicalRobustness2004}, is the ability to optimize internal parameters in response to external perturbations and adapt to new information to achieve the goals of an ongoing plan.}

<!-- While a robot may be able to move a cup of coffee to a precise location in space, its solution is often found to be brittle in a new context, or unable to generalize to the movement of new objects. We define a skill as a behavior that involves dexterity in Bernstein's sense. The use of a tool such as a screwdriver is an example of a motor skill. We define a task as the production of skilled movement in a particular context. Driving a screw in a particular posture using a particular screwdriver is an example of a task. These concepts will be further formalized in later chapters.  -->

<!-- Physiology is special for human hands so we chose them as a testbed -->

<!-- Human movement is ultimately the result of the activation and contraction of muscle fibers, and movements lie on a spectrum between reflexive and volitional. The supramuscular circuitry which determines the degree of volition we ascribe to movement, where volitional movement relies on supraspinal (though not necessarily conscious) processes. The human hand is a unique evolutionary invention that underlies our ability to perform various skills in a range of tasks-- movements that are decidedly volitional. It could be argued that the hand is in fact a crucial aspect of humanness. It is thought that the human cerebellar and neocortices evolved reciprocally to expand and support the computational burden of increasingly complex motor tasks such as tool-making and language production[REF???]. The hand is the pinnacle of dexterity and, as such, it is a fruitful testbed for studying the computations and circuitry that drive dexterous movement. A detailed physiological review of the hand and it's relation to skilled movement is described in ???.

<!-- Setting up a specific experiment will help us to track muscle-level changes -->

To explore dexterous movement, we will leverage recordings of muscles controlling the hand as a readout of flexible motor behavior. This is a step beyond recording hand kinematics as electromyography provides a physiological output of the nervous system. Surface electromyography recordings taken from the forearms controlling subjects' dominant hands allows us to track the sequential selection of muscle activations during both skill acquisition and subsequent performance of that skill to achieve desired goal. As we are interested in subjects' abilities to acquire new skills, we design tasks that require subjects to use available, but uncommon, motor activations. We then track the selection and execution of these activation during virtual tasks. Preliminary work in this direction is described in {@sec:experiment}.

<!-- we think theorizing with control and learning models will help us -->
<!-- what exactly from the theory world will help us? why is control/RL relevant? -->
<!-- what are we missing in the neuro / behavior lit that we need to borrow? -->
<!-- BE MORE SPECIFIC HERE -->

<!-- Admit that we need to collect and analyze data in an exploratory manner to then inspire hypotheses that can be modeled. We want these hypotheses to be inspired by theoretical work in control and learning theory -->

<!-- how are value computations connected to action and policy selection
how are feedback controllers adapted to motor errors, new environments, how are they learned as well as combined? -->

Using data from our experimental setup, we wish to understand both how the structure of muscle activation variability evolves during skill acquisition and how the motor system constructs skilled movement through the composition of component muscle activations. To begin, we review a sampling of current motor physiology research relevant to dexterous motor computations in {+@sec:physiology}. In {+@sec:experiment}, we cover our prototype hardware and experiments. With inspiration from physiology and our experiments, we hope to make progress in modeling sensorimotor control and learning in our experimental setup. We cover preliminary work in this direction in {+@sec:theory}, and discuss possible future directions in {+@sec:next_steps}.

<!-- To do so, we with the conceptual language of the experimental sensorimotor learning community together with the language of the control theory and reinforcement learning community, as each of these communities shares a common goal of understanding the computation underlying the production of skilled movement. -->


\subsection{Aims}

The overarching goal of this task is to track learning in a new movement contingency. Subjects have no prior knowledge related to the task, and must explore to find solutions, while experiencing constraints of their fitted decoder/task mapping. We want to follow their learning progress statistically to gain an insight into how subjects are learning to deal with their new environment.

Our theory of neural control of the hand is approximately: control is composed of a number of overlapping cortical controllers– these receive input from goal-oriented centers as well as a plethora of ongoing contextual, perceptual information. Control is thus modulated by these inputs, adjusting “online” to disturbance. Cortical controllers are massively redundant; the contain all available information about the context of an ongoing task, branching to an array of downstream spinal centers as well as converging to individual spinal innervations. Our hypothesis is that subjects will use their vast repertoire of pre-existing control schemes/movements/controllers/patterns/activations until they find a pattern that increases their success, upon which they will “hone” this scheme by refining the discovered movement. This hypothesis predicts an exploratory, or “search”, period of the task, followed by (or overlapping with) an exploitative or “honing” period as subjects settle on a motor solution. Our work is to highlight the statistical differences and attributes of these two sub-activities in our task, and explain how these activities relate to subjects’ natural hand movements and to theoretically optimal learning dynamics.

\cleardoublepage\printendnotes%
\ifSubfilesClassLoaded{%
    \newpage%
    \bibliography{../bib/bibliography}%
}{}%
\end{document}