\documentclass[../main.tex]{subfiles}
\graphicspath{{\subfix{../../images/}}}
\begin{document}

\chapter{CHAPTER NAME}\label{chap:LABEL}

\bigskip
\begin{quote}
\emph{QUOTE QUOTE QUOTE QUOTE QUOTE QUOTE QUOTE QUOTE }

\raggedleft{--- AUTHOR, \emph{SOURCE}}
\end{quote}

\cleardoublepage%

\section{GMM Null Space}

\begin{figure}[H]
  \centering
  \begin{minipage}{0.49\textwidth}
    \includegraphics[width=\textwidth]{more_results/gmms/nullspace_ratios_models.pdf}
    \subcaption{}
  \end{minipage}\\%
  \begin{minipage}{0.49\textwidth}
    \includegraphics[width=\textwidth]{more_results/gmms/nullspace_pvalues_models.pdf}
    \subcaption{}
  \end{minipage}
  \caption[Nullspace activity for GMMs]{CAPTION}\label{fig:nullspace_gmms}
\end{figure}


\section{Nullspace Ratio of Task Error}

\begin{figure}[H]
  \centering
    \includegraphics[width=\textwidth]{more_results/nullspace/example_error.pdf}
    \caption[Example Error]{CAPTION}\label{fig:example_error}
\end{figure}

\begin{figure}[H]
  \centering
    \includegraphics[width=\textwidth]{more_results/nullspace/error_ratio.pdf}
    \caption[Error Nullspace]{CAPTION}\label{fig:error_nullspace_subjects}
\end{figure}


\section{Nullspace Optimization Solutions}

Compare subject EMG solutions to optimization solutions
NB https://math.stackexchange.com/questions/2028698/
Hypothesis: subject error ratio of err-pinv / err-weighted will be > 1 as subjects tend to minimize their error to the prior-weighted solution rather than the pinv solution.
Todo – use a metric
Check if fitting different subsets of the prior data with GMMs makes the distance thing work out! 
Try calibration only
Try movement only
There is a result here regardless of the outcome!!!
This is about choosing a cost function and comparing to the data
Can I fit a cost function to the data? This is like fitting a model… 
We know subject solutions hit the target, but what do they minimize over time?
Can we regress these solutions against potential regressors?

How do subjects EMG solutions compare to optimization-derived solutions with varying levels of prior influence? I.e. Are subjects biased towards their prior data?

Do the variances of subject errors (subject solution - optimization solution) lie in the null space directions on average or not? I.e. Where do subjects expend most of their variance beyond achieving the requirements of the task 

Nonnegative min-norm solution - activity = mostly in the null space or task space? This says something about the error! Does the error live in the null space or task space, does this change over time?

\begin{figure}[H]
  \centering
    \includegraphics[width=\textwidth]{more_results/nullspace/computed_solutions_6.png}
    \caption[Computed solutions]{CAPTION}\label{fig:computed_solutions}
\end{figure}

\begin{figure}[H]
  \centering
    \includegraphics[width=\textwidth]{more_results/nullspace/distance_from_solutions.png}
    \caption[Distance from computed solutions]{CAPTION}\label{fig:computed_distances}
\end{figure}


\section{Classifying Nullspace Noise}


\begin{figure}[H]
  \centering
    \includegraphics[width=\textwidth]{more_results/nullspace/regression_scores.pdf}
    \caption[regression scores]{CAPTION}\label{fig:regression_scores}
\end{figure}

\begin{figure}[H]
  \centering
    \includegraphics[width=\textwidth]{more_results/nullspace/classification_scores.pdf}
    \caption[classification scores]{CAPTION}\label{fig:classification_scores}
\end{figure}

\begin{figure}[H]
  \centering
    \includegraphics[width=\textwidth]{more_results/nullspace/pca_2d_full.pdf}
    \caption[pca 2d full]{CAPTION}\label{fig:pca_2d_full}
\end{figure}

\begin{figure}[H]
  \centering
    \includegraphics[width=\textwidth]{more_results/nullspace/pca_2d_null.pdf}
    \caption[pca 2d null]{CAPTION}\label{fig:pca_2d_null}
\end{figure}


What do subjects learn? Do they learn specific rote movements, or are they flexible in their solutions?

Look at the null space for task relevant information-- think about this, have we removed all the task relevant information before looking at this noise?







\cleardoublepage\printendnotes%
\ifSubfilesClassLoaded{%
    \newpage%
    \bibliography{../bib/bibliography}%
}{}%
\end{document}