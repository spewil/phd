\section{Background Theory}\label{background-theory}

\{\#sec:bg\_theory\}

From the physiology, we see that the motor system is highly distributed
and constructs action based on a variety of state dependence. The
theoretical question becomes \emph{when does it make computational sense
to construct action by composing control policies rather than selecting
or tuning a single policy?} When is policy arbitration computationally
advantageous?

\textbf{The motor learning field does not yet possess an adequate
computational model for practice-induced increases in motor acuity.}
(Krakauer Motor Learning 2019)

\subsection{Existing Models of Motor Control and
Adaptation}\label{existing-models-of-motor-control-and-adaptation}

\subsubsection{Optimal Feedback Control}\label{optimal-feedback-control}

Stephen Scott review -- https://www.nature.com/articles/nrn1427 look at
the bullet points there, relate to our experiment

OFC is the best we got for motor coordination, but there's no adaptation
or learning

The control setup writes a cost, environment has some dynamics.

What is changing in this scenario? What is being learned? What
information is used to do this learning?

Which model variables correspond to muscles? Movements? What does the
resultant feedback controller compute? How does this relate to
cognition?

This model is lacking in \ldots{}

A key paper is \texttt{Valero-Cuevas\ 2009} which recording EMG from the
seven muscles driving the finger in a force-feedback task. The authors
found support for the ``minimum intervention principle''
{[}@Valero-Cuevas2009{]}.

\subsubsection{nonlinear iLQG models}\label{nonlinear-ilqg-models}

these are more predictive if we're actually using reaches experimentally

nagengast, braun, etc 2009
https://journals.plos.org/ploscompbiol/article?id=10.1371/journal.pcbi.1000419

\subsubsection{Noise in OFC}\label{noise-in-ofc}

\begin{itemize}
\tightlist
\item
  Nagengast 2010
  https://journals.plos.org/ploscompbiol/article?id=10.1371/journal.pcbi.1000857
  -- subjects are risk averse in the face of noise.
\end{itemize}

\subsubsection{Intuitive Example of the OFC
framework}\label{intuitive-example-of-the-ofc-framework}

Here we see a feedback controller with three muscles such that we can
plot the muscle activation trajectory.

This is the feedback controller K, we can understand it's action by
plotting

PLOT OF SIMPLE EXAMPLE

PLOT OF APT DATA

\subsubsection{Error-based Adaptation}\label{error-based-adaptation}

Error-based adaptation and state-space models have a great amount of
precedent in the sensorimotor learning literature. We will summarize
these models briefly and discuss our willingness to depart from them.

This is pretty much the best model we have that describe learning from
error

Current models of motor learning

x' = Ax + Bu

This model describes\ldots{}

The downsides of this model are that it describes a small aspect of our
data.

\subsection{Adaptive Linear Quadratic
Control}\label{adaptive-linear-quadratic-control}

Have to be careful about what is termed corrective, adaptive, and
learned.

\begin{quote}
Mathematically, we can formalize an adaptive control problem as a
mapping x t+1 = F(x t , u t , a) with unknown system parameters a that
have to be estimated simultaneously with the control process (Sastry and
Bodson, 1989; Åström and Wittenmark, 1995). (Braun 2009\ldots{} Wolpert)
\end{quote}

\begin{quote}
In the following we will refer to changes in the control policy that
occur within individual movements as ``adaptation'' to distinguish them
from ``learning'' processes that improve these adaptive responses across
trials. (Braun 2009\ldots{} Wolpert)
\end{quote}

\begin{quote}
Online correction is, for example, required in the case of an
unpredicted target jump. Under this condition the same controller can be
used, i.e., the mapping from sensory input to motor output is unaltered.
However, unexpectedly changing the hand--cursor relation (e.g., by a
visuomotor rotation) requires the computation of adaptive control
policies.
\end{quote}

\begin{quote}
Strictly speaking, an adaptive control problem is a nonlinear control
problem with a hyper-state containing state variables and (unknown)
parameters. This means in principle no extra theory of adaptive control
is required. In practice, however, there is a well established theory of
adaptive control (Sastry and Bodson, 1989; Åström and Wittenmark, 1995)
that is built on the (somewhat artificial) distinction between state
variables and (unknown) parameters. The two quantities are typically
distinct in their properties. In general, the state, for example the
position and velocity of the hand, changes rapidly and continuously
within a movement. In contrast, other key quantities change discretely,
like the identity of a manipulated object, or on a slower timescale,
like the mass of the limb.
\end{quote}

We're really interested here in the learning problem! And how we can
test and model this within the framework of OFC.

\subsection{Motor Adaptation}\label{motor-adaptation}

Implicit / Explicit Model-based / Model-free Slow / Fast

Krakauer et al.'s categorization of motor learning places prior work
into the following classes: - Adaptation - Sequence Learning - De Novo
Learning - Motor Acuity - Expertise

work in reaching shadmehr, krakauer reviews

\subsubsection{State-space Models of Motor
Adaptation}\label{state-space-models-of-motor-adaptation}

\emph{Modeling Sensorimotor Learning with Linear Dynamical Systems} by
Cheng and Sabes, 2006. The goal is to model trial-by-trial learning by
fitting data to a linear dynamical system model. Here we'll call \(F_t\)
the \textbf{sensorimotor mapping} transforming inputs \(w_t\) to \(y_t\)
outputs per trial:

\[
y_t = F_t(w_t, \gamma_t).
\]

This can be thought of as a mapping from inputs within a single trial
to, for example, endpoint error. Noise is captured by the \(\gamma_t\).
The trajectory in \(F\) space attempts to capture the process of
learning. The learning rule \(L_t\) can be written

\[F_{t+1} = L_t\left(\left\{F_\tau\right\}_{\tau=1}^{t}, u_t, \eta_t, t\right)\]

where \(\left\{F_\tau\right\}_{\tau=1}^{t}\) is the history of the
mapping, \(u_t\) is the history of the total inputs to learning which
could encompass \(y\), \(w\), and exogenous inputs \(r\). Noise in the
learning is captured by \(\eta\).

We can approximate this learning problem using linear equations by
assuming that \(L_t=L \ \forall \ t\) is stationary, \(F_t\) is
parameterized by \(x_t\in\mathbb{R}^y\). Thus,

\[
\begin{aligned}
y_t &= F(x_t, w_t, \gamma_t) \\
x_{t+1} &= L(x_t, u_t, \eta_t).
\end{aligned}
\]

The trial-to-trial input-output mapping \(F\) is now fixed, and is
transformed by trial through its parameters \(x_t\) by \(L\). Note that
both mappings are Markovian and there are two input vectors, one for
within-trial and one between-trial. These can include overlap. We can
now linearize these mappings around an equilibrium point:

\[
\begin{aligned}
x_{t+1} - x_e &= A(x_t-x_e) + B(u_t-u_e) + \eta_t \\
y_t - m_e &= C(x_t-x_e) + D(w_t-w_e) + \gamma_t
\end{aligned}
\]

As shown by Cheng and Sabes, we can bundle the equilibrium terms into a
bias term and drop this term if we mean-subtract our data
(\(x_t, y_t, u_t, w_t\)) when it's time to fit. This gives us a simple
linear dynamical system:

\[
\begin{aligned}
x_{t+1} &= Ax_t + Bu_t + \eta_t \\
y_t &= Cx_t + Dw_t + \gamma_t.
\end{aligned}
\]

The first equation governs the evolution of parameters of the
within-trial input-output mapping, while the second equation governs the
trial output given the current within-trial mapping parameters \(x_t\)
and learning inputs \(w_t\). The parameters \(x_t\) are hidden variables
that are only observed through the output \(y_t\). The noise terms
\(\eta_t\) and \(\gamma_t\) are normally distributed with covariances
\(Q\) and \(R\), respectively. \(A\) governs the passive trajectory of
\(x_t\). If \(A=\mathbb{I}\), \(x_t\) does not decay passively.

There is a general form for this model which separates endogenous input
\(y_t\) from exogenous input \(r_t\)

\[
\begin{aligned}
x_{t+1} &= Ax_t + [G \ H][r_t \ y_t]^T + \eta_t \\
y_t &= Cx_t + Dw_t + \gamma_t
\end{aligned}
\]

where \(H\) governs biases in directions of the outputs. A unbiased
output is isotropic. To add explicit stationary bias we write

\[
\begin{aligned}
x_{t+1} &= Ax_t + Gr_t + Hy_t - Hb_x + \eta_t.
\end{aligned}
\]

\subsubsection{Example Models}\label{example-models}

\paragraph{Feedback Error Learning}\label{feedback-error-learning}

\[x_t+1 = Ax_t + [H\ H][-y_t^*\ y_t]^T\]

The second term is simply the difference between the output \(y_t\) and
the desired output \(y_t^*\).

\paragraph{Prediction Error Learning}\label{prediction-error-learning}

Let \(u_t = y_t - \hat{y}_t\) where \(\hat{y}_t\), the difference
between the output and the predicted output such that

\(\hat{y}_t = Cx_t + Dw_t\). Thus,\(\hat{y}_t\) is a kind of forward
model. Plugging in,

\[x_{t+1} = Ax_t + Bu_t + \eta_t\]

becomes

\[
\begin{aligned}
x_{t+1} &= Ax_t + B(y_t - Cx_t - Dw_t) + \eta_t \\
x_{t+1} &= (A-BC)x_t + By_t - BDw_t + \eta_t
\end{aligned}
\]

\paragraph{Target Prediction Error
Learning}\label{target-prediction-error-learning}

Now let \(u_t = \hat{y}_t - y^*_t\), the difference between predicted
output and target output.

\[
\begin{aligned}
x_{t+1} &= Ax_t + B(Cx_t + Dw_t - y^*_t) + \eta_t \\
x_{t+1} &= (A+BC)x_t + BDw_t - By^*_t + \eta_t
\end{aligned}
\]

\paragraph{Steady State}\label{steady-state}

If we take the output and state vectors in expectation for constant
inputs \(w\) and \(r\), we have

\[
\begin{aligned}
y_\infty &= \lim_{t\to\infty}\mathbb{E}\left[Cx_\infty + Dw\right] \\
x_\infty &= \lim_{t\to\infty}\mathbb{E}\left[Ax_\infty + Bu\right] \\
&= Ax_\infty + Gr + Hy_\infty \\
&= Ax_\infty + Gr + HCx_\infty + HDw \\
-(A + HC - \mathbb{I})x_\infty &= HDw + Gr.
\end{aligned}
\]

Thus, the eigenvalues of \(A + HC\) must be less than or equal to 1 for
\(x_\infty\) to be stable in expectation.

\subsubsection{Critique}\label{critique}

\begin{quote}
It should be emphasized, however, that these models are not intended to
provide a mechanistic explanation of adaptation---they do not explain
why adaptation has the properties it does. They explain neither why
compensation for a perturbation decays, nor why people learn at the rate
they do. However, these models do encapsulate a set of simple
assumptions about how learning might occur on a single-trial timescale,
and allow us to predict behavior in response to sustained or fluctuating
perturbations over many trials. (Krakauer)
\end{quote}

\begin{quote}
{[}Bayesian theories of learning{]} hold that adaptation is essentially
a problem of estimating the properties of the imposed perturbation,
given uncertainty about sensory feedback and the state of the world.
Mathematically, under certain assumptions (that the noise/variability is
Gaussian in both cases), this Bayesian estimation framework becomes
equivalent to a Kalman filter (219)---a common algorithm for optimally
tracking dynamic states under noisy observations--- which is almost
identical to a state-space model. (Krakauer)
\end{quote}

\subsection{Two-rate models}\label{two-rate-models}

\[
\begin{aligned}
X_{t+1} &= X^{s}_{t} + X^f_t \\
X^s_{t+1} &= L_s \cdot e_t + R_s \cdot X^s_{t} \\
X^f_{t+1} &= L_f \cdot e_t + R_f \cdot X^f_{t} \\
\end{aligned}
\]

where we fit \(L_i and R_i\), the learning rate and retention
parameters. (shadmehr 2006)

\begin{quote}
Observations have revealed that there is far more to how participants
compensate for an imposed perturbation than just implicit recalibration
of a pre-existing motor controller. Instead, multiple, qualitatively
different processes occur during adaptation tasks; for example,
processes driven by explicit, cognitive strategies. When it comes to
studying implicit recalibration, these other processes can be a
contaminant. At the same time, however, these additional processes
likely reflect the involvement of similar mechanisms to those
responsible for more general motor skill learning. (Krakauer 2019 Motor
Learning Review)
\end{quote}

\begin{quote}
it is unlikely that the underlying components that contribute to
learning in adaptation paradigms only differ in terms of their learning
and retention rates, as the two-state model suggests. The multiple
components of learning instead correspond to entirely distinct learning
processes that are simultaneously brought to bear on the same problem.
(Krakauer 2019 Motor Learning Review)
\end{quote}

% <!-- The UCM is not a hard-and-fast principle, as nothing as in the motor system. Rather, as we've seen elsehwhere, there seems to be a spectrum of control. This could be explained through a composite cost function which penalizes deviations from prior movement strategies[@raczSpatiotemporalAnalysisReveals2013]. There is much research pushing back on optimal control, uncontrolled manifold hypothesis, and this will be addressed in {+@sec:experiment}. -->


% <!-- Since the value function represents cost-to-go, it would be sensible to move down this landscape as quickly as possible. Indeed, is in the direction of steepest descent of the value function. However, not all directions are possible to achieve in state-space. represents precisely the projection of the steepest descent onto the control space, and is the steepest descent achievable with the control inputs . Finally, the pre-scaling by the matrix biases the direction of descent to account for relative weightings that we have placed on the different control inputs. Note that although this interpretation is straight-forward, the slope that we are descending (in the value function, ) is a complicated function of the dynamics and cost. (Tedrake http://underactuated.mit.edu/lqr.html)-->

% <!-- A solution to the algebraic Riccati equation can be obtained by matrix factorizations or by iterating on the Riccati equation. One type of iteration can be obtained in the discrete time case by using the dynamic Riccati equation that arises in the finite-horizon problem: in the latter type of problem each iteration of the value of the matrix is relevant for optimal choice at each period that is a finite distance in time from a final time period, and if it is iterated infinitely far back in time it converges to the specific matrix that is relevant for optimal choice an infinite length of time prior to a final period—that is, for when there is an infinite horizon.  wiki riccati equation-->

% <!-- The unknown mapping $M$ from muscle to task space looks like the observation matrix $H$ in the LQG problem:

% \begin{align*}
% y_t &= Hx_t + v_t\,\,\,(\mathrm{LQG}) \\
% y_t &= Mx_t + v_t. \,\,\,(\mathrm{experiment})
% \end{align*}

% The state dynamics in the task are of the form:

% \begin{align*}
% x_{t} &= Ax_{t-1} + Bu_{t-1} + w_{t-1} \,\,\,(\mathrm{LQG}) \\
% x_t &= Dx_{t-1} + Iu_{t-1} + w_{t-1} \,\,\,(\mathrm{experiment})
% \end{align*}

% where $D$ is a diagonal decay matrix of with terms $\mathrm{e}^{-\lambda}$ and $I$ is the identity. The subject produces muscle contractions which add to the current latent (unobserved) state. In the absence of control signals, the state decays back to $0$ in line with the physics of your arm returning to a passive state in the absence of muscle contractions. The terms $w$ and $v$ are gaussian noise vectors with distributions $\mathcal{N}(0,Q)$ and $\mathcal{N}(0,R)$. If we combine the transition and observation models:

% \begin{align*}
% y_t &= MDx_{t-1} + Mu_{t-1} + Mw_{t-1} + v_t \\
% &= A^\prime x_{t-1} + B^\prime u_{t-1} + Mw_{t-1} + v_t.
% \end{align*}

% We can think of this as the combined system identification problem where $A^\prime=MD$ and $B^\prime=M$ are unknown and must be estimated. The noise covariances of this altered system are now non-trivial, however. We could also assume that the transition dynamic $D$ is known and that the identification problem is learning the mapping $M$ only. This might not be a poor assumption since the exponential decay is meant to serve as an intuitive passive dynamic.

% In each trial of the task, a subject will have some internal representation of the observation dynamic $M$ which may or may not be accurate. In order to make accurate predictions, $M$ must be estimated.

% Learning linear dynamical systems from data is a hot topic of research, most of which seems to focus on learning in the context of complete state observation ($M=I$, $y=x$). Algorithms to determine parameters of $A$ and $B$ are proposed (see Dean, Recht 2018).

% From LQG theory we know that the control law is a linear function of the state:

% \begin{align*}
% u_t = -L_tx_t
% \end{align*}

% and thus our combined system dynamic is:

% \begin{align*}
% y_t &= M(D-L_t)x_{t-1} + Mw_{t-1} + v_t.
% \end{align*}


% The noise covariance due to the observation Q is unchanged, but the new noise covariance for the latent process is now $MRM^T$. This may make things difficult. -->